%************************************************
\chapter{Proposed Work}\label{ch:mathtest} 
%************************************************
\section{Introduction \& Related Works}

Datasharing collectives hold promise for workers to 1) gain understanding of self and the collectives so as to enable goal-setting and self-improvements 2) hold platforms and customers/clients accountable for responsible (as well as minimal) data collection practices and 3) collectively shape and propel policy decisions related to their working conditions. In prior literature, scholars compile evidence for the effectiveness of tech-enabled collective organization for providing strategic value (i.e. information sharing, collective resource-mobilisation, networking and (response-)generation) in moments of community crisis -- i.e. disasters \citep{selforg}. 
In domains of social media (e.g., WhatsApp Explorer \citep{whatsapp}, Instagram Data Donation \citep{insta}, TikTok Social Media Donator \citep{tiktok}) and healthcare (e.g., Health Data Exploration \cite{hde}, personal data donations in Germany \cite{pddGermany}), previous investigations leveraged data donation tools to explore user engagement (of teens), effects of social media on elections/collective violence, as well as opportunities for health-focused citizen science. Such studies of data donation tools highlight ongoing issues, which include: motivations for sharing \cite{hde, willingness}, issues of access (e.g., privacy \cite{best}, ethics \cite{willingness}, bias \cite{nonparticipation}, accessibility \cite{health} and consent \cite{port, osd2f}), as well as data quality \cite{pddGermany} and security \cite{osd2f}. While some of these previous tools provide guidelines for establishing guardrails that define rules on how researchers should preprocess and filter donated data before analysis \cite{module}, none explored the possibility of allowing workers to define and construct their own data types/initiatives -- an action that can affords users further agency, fosters collective identity, and offers users an avenue to collaboratively influence policymaking.

In the context of gig work, prior works explored potential infrastructures for collective data institutions \cite{steinyou}, designed sousveillance tools to counter platform/customer surveillance \cite{sousveillance}, and considered how data collectives can advance relevant regulation \cite{collectives}. Meanwhile, the research community increasingly call for more qualitative engagements (as opposed to strictly quantitative empirical measurements) with (worker) communities \cite{talkfutures, diary}.

In our initial evaluation of Gig2Gether (Chapter X), workers expressed desires to 1) collectively understand platform functionalities by exchanging experiential information and 
2) gather additional (and evolving) data types to gain a better grasp of how platform algorithms operate. Thus, we propose the following work to address the needs workers expressed for diversified data types (across various gig work domains and platforms), as well as to accommodate the ever-evolving nature of platform algorithms (and related data they collect).

% three motivations prompting workers to engage with the datasharing tool: (1) potentials for workers to improve self or others through strategy-learning or sharing (2) desires to 
Goal 1: \textbf{Worker-Defined Data Types \& Initiatives}. Develop methods for workers to define data forms/units/types for sharing with the community, so as to 
\begin{enumerate}
    \item enable worker-led documentation and case-building of unnecessary data collection --- policymaking \& litigation purposes
    \item keep platforms accountable for over-surveillance or privacy violations, and encourage practices data minimalism --- informed collection of only strictly necessary data for business purposes --- accountability \& sousveillance purposes
    \item support workers in understanding and resisting practices that harm worker well-being (e.g., uninformed pay cuts/nonpayments, deactivations, discrimination, compromises of safety/privacy) --- self-tracking, tax \& data visualization purposes
    \item help stakeholders set and achieve work/policy goals --- self/collective-improvement purposes
\end{enumerate}
\rule{\linewidth}{0.5mm}

Beyond a sustainable workflow that tackles the diverse and ever-evolving nature of platforms, refined controls for privacy and consent might also motivate workers to engage in data contributions. Prior works revealed a plethora of privacy concerns that gig workers shared in online forums, including issues with the overt (and often covert) methods of data collection by both platforms and clients -- leading to feelings of surveillance, as well as worries around asymmetrical amounts of information disclosures they must provide as compared to customers \cite{privacy}. In these online forums, researchers observed qualitative forms of narrative sharing through advice-seeking and -giving, as well as  instances of venting/ranting/commiseration \cite{privacy, peersupport} -- phenonmena that we also observed on the Stories feature of Gig2Gether. While we intend to continue supporting such interactions, we must recognize and address the discomfort and privacy concerns associated with disclosing personal information with others (present even for collaborating members of the same team \cite{toshare}), especially since 1) such discomfort cause individuals to avoid seeking help \cite{avoidance}, and 2) self-disclosures directly impact well-being through various mechanisms \cite{disclosurewell, psychological}.

% \begin{itemize}
%     \item inherent tradeoffs between privacy and bias
% \end{itemize}

To motivate voluntary and consensual self-disclosures, prior works found various factors to encourage higher levels of self-disclosures, including visual anonymity \cite{awareness}, avatars that deviate from user appearances \cite{avatar}, as well as self-efficacy and the upkeep of a professional online images on social media \cite{predictors}. For self-disclosures that particularly sought help from others, researchers found that features such as anonymity, ephemerality and system routing help reduce social costs for the answer-seeker \citep{socialcost}, while meronymity (where people select identity signals to reveal) encourages newcomers to digitally solicit advice from their seniors \cite{meronymity}. 

Goal 2: \textbf{Granular Control and Consent for Self-Disclosures.} 
    \begin{enumerate}
        \item Offer workers a range ways and degress (e.g., pseudonymity, meronymity, full anonymity) of preserving anonymity when sharing personal work information (including profile-level demographics and experiential knowledge) to minimize (social) costs of help-seeking 
        \item Employ privacy-preserving/enhancing frameworks such as differential privacy or cryptography
    \end{enumerate}
\rule{\linewidth}{0.5mm}
Motivation here draws from (1) Zheng's thesis pointing out lack of organization/governance with existing forums, (2) related literature on collective sense/decision-making (for e.g., terms of privacy) and (3) PolicyCraft.

Goal 3: \textbf{Moderation, Group Disclosures \& Communication Structures for Multi-Stakeholder Policy Development.} Provide
    \begin{enumerate}
        \item Mechanisms/spaces for workers to collectively construct terms of privacy or ground rules for interaction, with considerations how these rules may change over time
        \item Methods for organizing information and cross-stakeholder communication to support collective information sharing and decision making around policies related to platform-based labor
    \end{enumerate}

\section{Research Questions}
\textbf{RQ1}: To what extent can workers abstractly describe desired data types to contribute, for supporting (collective) sensemaking/decision-making around relevant policy initiatives, in a manner that does not require upload of sensitive personal data?  \\
\textbf{RQ2}: Leveraging worker contributions (of abstract data types/initiatives), what methods of data self-disclosures, cross-stakeholder communication and collective decision-making most effectively
\begin{enumerate}
    \item worker incentives for self-disclosures/data contributions
    \item collaborative, multi-stakeholder policy deliberation and
    \item practical impact on policy-making?
\end{enumerate}
\section{Proposed Design of System Extensions}
\begin{itemize}
    \item Worker-constructed data initiatives on 1) data they know/hypothesize that platforms collect 2) information they desire to collect about platforms/customers for conducting sousveillance and 3) self-tracked data to keep personal informed (for purposes such as taxes) --- so as to gather evidence for proposing new policies for further regulations of platforms. Each piece of worker-contributed data initiative can be labeled with
    \begin{enumerate}
        \item involved data types (screenshot should include pay rates/receipts)
        \item whether the data should be collected (is it critical for business purposes)
        \item who should have access to the data type, in addition to platforms (workers alone, policymakers, general audience)
    \end{enumerate}
    \item Customizable disclosure preferences over shared profile information (options include pseudonymity, anonymity, meronymity)
    \item Agency to choose ephemerality of interactions and contributed data
    \item Ability for workers to collectively define community norms for moderation, governance and privacy
\end{itemize}

\section{Methods}
\begin{enumerate}
    \item Survey existing literature for strategies on community data donations \& initiatives, as well as methods of protecting individual and group-level worker privacy
    \item Develop initial mock-ups of ideas for proposed system extensions 
    \item Iteratively refine prototypes of system extensions with pilot workers
    \item Conduct a between-subjects study with 2+ conditions to evaluate how new methods affect workers' motivations to contribute or levels of comfort with self-disclosures
\end{enumerate}

\section{Timeline}