\chapter{Gig2Gether: Cross-Platform Datasharing to Unify \& Demystify Workers} \label{6gig2gether}
In Chapter \ref{support}, we ideated with workers and policy experts to identify relevant initiatives that datasharing can support, as well as practical considerations to guide the development of datasharing systems that effectively exchange information among stakeholders for advancing collectivism and policy.
In the HCI and CSCW communities more broadly, the recognized need to share information and build solidarity among gig workers has prompted calls toward worker-centered data collectives \cite{supporting, workshop, end, uuapp, sousveillance}. 
However, most gig workers lack practical experience with intentionally contributing personal data for purposes of building collectivism or informing policy initiatives, leaving an open question of whether workers themselves would be motivated to engage in data-sharing. Despite existence of prior systems that focused on building collectivism through data/experience-sharing within specific gig work communities \cite{dynamo, 6B4U}, we are unaware of research exploring how workers engage with a cross-platform data-sharing tool situated within their everyday workflows.

{In this chapter, I describe how we took an initial step to address this gap by building and evaluating a prototype data-sharing system aimed to connect active workers from three gig platforms.
Based on design requirements {(\S\ref{Related_Work_Design})} derived from related literature, we constructed early wireframes and subsequently
conducted pilot testing with workers in our target domains to ensure alignment with worker preferences and refine usability.}
{This multi-stage design process culminated in} Gig2Gether (\S\ref{gig2gether}), a {prototype} system {enabling} US-based workers {from three gig platforms} to (1) {engage in cross-platform mutual support through data- and experience-sharing, so as to promote larger-scale solidarity and cooperation} (2) track {and reflect on experiences and statistics that report on aggregated and specific experiences of subpar working conditions}
 as well as (3) {strategize and} plan {for improving their gig careers}. {Beyond} {surfacing possible cross-platform} worker {interactions, reactions and unfulfilled }desiderata, Gig2Gether is intended to eventually serve as a portal {for exchanging knowledge, insights and resources} between workers, advocates and policymaking experts.

Through a subsequent field study {(\S\ref{field})} with 16 gig workers across three platforms/work domains, we {surfaced three main themes around how gig workflows can integrate data-sharing for empowering collectivism and advancing policy.} 
{First, the exchange of experiential strategies and challenges allowed workers to engage in cross-platform mutual aid, 
individual tracking of financial data enabled them to reflect on and plan work,
while potential shared tracking of  aggregated statistics helped them imagine use cases of collaboratively reasoning about platform mechanisms and rates. Second, workers expressed willingness to share both aggregated statistics and qualitative accounts of lived experiences with other stakeholders, for purposes of helping inform policy creation, especially around issues of safety and wages. Third, we overview how data-sharing integrated into workers' varied workflows, describing practical challenges that inform desires of future affordances as well as requests for additional metrics and data.}
Finally, we discuss and reflect on {new and foreseeable practical challenges unveiled (\S\ref{ongoing})}, potential implication for {advocacy and policy influence (\S\ref{door}), as well as ways that data-sharing can complement existing and alternative means of worker empowerment (\S\ref{complement})}.

\section{Related Work}\label{Related_Work}
{Recent efforts} coalesced around 1) the potential of worker data for making evident the conditions imparted by algorithmic platform practices and 2) the importance of concrete policy and regulation that ensure strong worker protections. 
Below, we describe how related works center our vision {and design} of a worker-centered data-sharing platform to meet {needs of workers (across platforms)} for self-tracking, mutual aid, and policy advancements.

\subsection{{Demands of Gig Work Across Platforms}}\label{Related_Work_Challenges}

{A burgeoning body of work investigating} gig work surfaced challenges emerging from {platform-based gigs}{, but most of these studies examined issues with respect to a specific platform, thereby revealing challenges that apply to only one domain of work --- e.g., safety hazards in ride-hailing and food delivery \cite{nilsenhealth, stressfulride, healthdrive, deliverysafe}, wage theft in care work contexts \cite{ming2024wage, cole2024wage, jerseycare}, or irregular schedules in online freelancing \cite{sousveillance, personal, freelancecontrol}. Such platform-specific focuses limit insights on whether uncovered stressors generalize to other contexts --- similarities in such experienced challenges can serve to unify workers across platforms. Interventions for unifying gig workers is especially necessary since they often do not self-identify as gig workers, and instead use platform-specific terms (e.g., Uber driver) to self-describe \cite{supporting}. In the few cases where multi-platform analyses were conducted, studies revealed how platforms shared higher-level risks (e.g., privacy, financial, psychological, gender biases) \cite{privacy, toward, ma2022brush} --- most of which benefit from further discourse, reporting and exposure --- collective actions that a data-sharing tool can help facilitate.}

{While the burdens of gig work surface differently across platforms, underlying causes of work challenges are often similar: a lack of labor/safety standards and regulation gives way to unbridled worker exploitation through algorithmic management \cite{dubal2023algorithmic, machines, excessive}, gamification tactics and information asymmetries \cite{algorithmic, locus, zhang2022algorithmic}, and an absent collective worker voice that stifles public awareness of harms \cite{ming2024wage, cole2024wage, lastmile}.}
In the {US}, gig workers are typically classified as independent contractors -- resulting in limited policy or regulatory protections over work conditions, {making} workers {compulsory} to manag{ing} a bevy of {logistical} obligations related to self-employment{: fulfilling tax requirements \cite{taxing, tax, tax_lives, returns} through self-}tracking {of} earnings and expenses \cite{accountable, taming}, {conducting unpaid labor to find, procure and scope gigs in times of precarity \cite{youth, france, freelancecontrol}, assuming costs of work-induced injuries (in lieu of workers' compensation and health insurance) \cite{nilsenhealth, healthdrive, deliverysafe, codesign}, managing psychological costs to
working alone \cite{atom, alienated, commodified}, and so on.} {Such similarities in overarching causes of challenges to gig work suggests an opportunity for technology interventions to build solidarity between the currently fragmented and scattered worker communities.}

\subsection{{Individual Tracking \& Sensemaking $\rightarrow$ Mutual Aid \& Collective Decision-making}}
In the absence of peer support and higher power actors who assume or share the structural risks and challenges inherent to gigs, workers are left to their own devices to manage {various} accountabilities {and obligations} \cite{consent, dalal2023understanding, accountable}. 
Studies documented two main ways that workers understand and manage work: on their own through self-tracking, or with peers via online groups/forums. 
Recent work at the intersection of HCI and Personal Informatics revealed how gig workers currently (or might in the future) self-track to (1) protect themselves from the platform \cite{privacy} or customers \cite{visibility, sousveillance} (2) comply with tax obligations \cite{tax_lives, taxing} (3) understand how algorithms operate \cite{sousveillance, zhang2023stakeholder} and (4) comprehend and improve their own earning patterns \cite{zhang2023stakeholder, accountable, supporting} using tools such as data probes {in addition to apps designed for tracking fuel, time, tax, mileage \footnote{Tracking apps include Fuelio and GasBuddy (for Fuel), Traqq (for time), Stride (for tax), as well as MileIQ, Everlance and Triplog (for Mileage)} and generalized gig work assistance \cite{accountable}. For instance, Mystro a commercial tool affording rideshare drivers the agency to auto-decline work across platforms that do not match their expressed preferences (e.g., earning rates, duration of gigs, work locations). 
Gridwise and Farepilot provide workers data-driven insights about in-demand locations, while Stride assists with tax filing. } 

While such tools act as resource providers and (sometimes automated) assistants, they {fall short in} provid{ing} workers with social support or strategies in times of need. Thus, to overcome {the atomized nature gigs \cite{peersupport, atom}} and find a sense of   ``community'',  workers also leverage online forums (both {pages and groups on general-purpose sites like Reddit/Facebook} and platform-specific sites like uberpeople.net) to share strategies \cite{nomad} and information \cite{machines, peersupport} {so they can hypothesize and collective make sense of underlying platforms' algorithmic mechanisms}, solicit advice and social connections \cite{belonging, organizing}, as well as rant and commiserate \cite{privacy, atom}.  
In addition, online video tutorials (e.g., vlogging) are emerging as a more effortless way for workers to learn about existing strategies and work conditions \cite{chan2019becoming, pires2024delivery, woodside2021bottom}.
However, the loosely-organized structure of general purpose forums (and video sharing platforms) {makes them in}effective for sensemaking \cite{peersupport}, while platform-specific sites limit {worker's abilities to discern unifying challenges shared across domains from characteristics that uniquely afflict workers of a single work context/platform}.
{Furthermore, ``Online forums are built to aid workers with a sense of immediacy, not to quantifiably or qualitatively monitor
request patterns or worker grievances over time'' \cite{organizing}, making them ill-suited for purposes of collective bargaining or identify-building.}


\subsection{Gaps in Current Approaches toward Worker-Centered Datasharing for Policy Change}\label{Related_Work_Using_Data}

Recognizing the potential of worker data to support workers' sensemaking and auditing of platform algorithms, researchers and worker groups increasingly turn to worker data exchange tools as a means to {unify and} empower gig workers.  
{Taking a first step in considering how data collectives can mitigate information asymmetries of gig work, }\citet{uuapp} {used participatory design to deliberate on} variants of data collection institutions with rideshare and delivery drivers, exploring methods of \textit{data leverage} (also covered in \cite{levers}), \textit{governance structures}, and \textit{access control}.
Through the lens of care ethics, \citet{sousveillance} used mockups with Upwork freelancers to uncover needs for relieving emotional strain, finding legitimate gigs, and measuring/managing invisible labor; they also surfaced the importance of ensuring that newly created sousveillance tools\footnote{whereby workers (under surveillance) monitor those in positions of authority (i.e. requesters/platforms) by collecting data about them} do not generate additional invisible labor for workers. 
\citet{6B4U} partnered with a delivery worker organization to {build} the Shipt Calculator, which {audits the platform's wage determination algorithm by allowing workers to} share work wage data and subsequently aggregat{ing it to calculate changes in} commission rate{s}. 
{Related} organizations (e.g., Worker Info Exchange\footnote{https://www.workerinfoexchange.org/} and Worker's Algorithm Observatory\footnote{https://wao.cs.princeton.edu/}) formed to help platform workers collect data and investigate algorithmic decisions. 
To help {workers} explore and contextualize {surfaced} data patterns with their positionality, well-being, and experiences, \citet{zhang2023stakeholder} create{d} data probes---interactive visualizations from {Uber driver}'s data. {To identify where worker needs meet feasible policy changes, Hsieh and Zhang et. al.} \cite{supporting} interviewed policy domain experts and conducted co-design workshops with gig workers to {understand their shared and distinct priorities, in addition to how data can help meet such objectives. While these works surfaced key design requirements for envisioned data-sharing tools, the lack of a working prototype functioning under realistic working conditions constrains the degree to which such studies can identify and confirm the concrete and practical desires and challenges of workers when integrating such hypothesized data-sharing systems into their everyday workflows}.

{Among systems that leveraged data to support platform-based workers in building collective bargaining power, a few were employed and embedded into everyday workflows of platform-based laborers:}
the recent Shipt Calculator {solicited pay data from workers via a SMS bot to measure aggregate changes in different rollouts of the pay algorithm} \cite{6B4U} while {the seminal work of} Turkopticon {solicited worker-contributed ratings of crowdwork requesters to surveil and hold them accountable from below (i.e., sousveillance)} \cite{turkopticon}; follow-up work with Dynamo {directly solicited ideas for action around issues surrounding labor in Mechanical Turk, and subsequently supported workers to form publics and mobilize towards action for such ideas} \cite{dynamo}{. By working with only quantitative data around pay, the Shift Calculator constrained understandings of worker struggles to a singular data type. While Turkopticon and Dynamo help unify workers to surface a diverse set of issues for impacting policy changes, the study context focused narrowly on a platform for online crowd work --- the challenges of which diverge significantly from those afflicting workers of other gig work contexts, especially those who perform labor offline.} 

Finally, prior works explored how to leverage worker data to advance driver-centered policies. \citet{parrott2018earnings} published a formative economic analysis {of} working conditions and wages of drivers in New York City {using Uber and Lyft app data}, subsequently proposing a minimum wage standard {for drivers} that was adopted in the city. This data-driven strategy to assess the need for a driver minimum wage policy has been replicated in Seattle \cite{reich2020minimum} and Massachusetts \cite{jacobs2021massachusetts}. Non-profits and other researchers also followed this template on smaller scales, {using} data from worker surveys {rather than app data (due to data access restrictions)} \cite{leverage2022gig, McCullough_Dolber_Scoggins_Muna_Treuhaft_2022, McCullough_Dolber_2021, washington2019delivering}. 
In follow up work to \cite{zhang2023stakeholder}, \citet{policy_probes} explored how workers' data can support policymakers and policy informers, surfacing the{ir} potential to 1) inform policy creation, 2) support lobbying efforts, 3) help worker organizations grow member strength, 4) aid regulatory efforts \footnote{One example is a nascent regulatory effort around \href{https://www.ftc.gov/business-guidance/blog/2024/03/price-fixing-algorithm-still-price-fixing}{algorithmic pricing investigations}}.
To complement these previous approaches that aggregated quantitative data, this work aims to facilitate information exchange and collaboration between worker communities and supporting stakeholder groups so as to bridge fragmented worker communities and simultaneously advance policy. Further, we strive to develop such mechanisms in a way that highlights key insights and context on critical work issues such as safety and discrimination, as outlined Chapter \ref{support}.

\section{Design Requirements for Worker-Centered Data-sharing Tools}
\label{Related_Work_Design}
{ 
While we recognize the populations of workers who complete gigs but do not use gig platforms to procure them --- e.g., contractors belong to LLC's or other small businesses, as well as artists or musicians who leverage other means of networking to acquire gigs --- we do not consider such groups to be under the scope of this study, since their job acquisition process do not require individual workers to interact with a gig platform as an algorithmic intermediary.}

% \setcounter{subsection}{-1}
\subsection{{Design Requirement 1}: Center Worker {Needs \&} Goals {to Advance Policy Initiatives}} \label{DR1}
{Most related studies exploring designs of tools for building collectivism through data contributions approached the issue with worker-centered and participatory design methods \cite{uuapp, zhang2023stakeholder, supporting}. Several of such studies suggested that identifying and accommodating diverse worker needs requires} workers to share both statistical and contextual data around their working conditions {\cite{individualized, policy_probes}}, in a way that meets their current individual goals and workflows, while ensuring they retain agency (i.e. data control) over what they share, how often they share, and who they share to {\cite{supporting, uuapp}}. In the following three sub-requirements, we detail how {such works surfaced needs} to respect existing habits and preferences of workers while supporting data contributions that promote self-assessment and policy advancements.

\subsubsection*{\tcbox[colframe=lightgray, colback=lightgray]{DR 1.1}: \textbf{Support Quantitative and \uline{Qualitative} Data Sharing} {for Impacting Policy}}

\label{DR1.1}
As \ref{Related_Work_Challenges} details, gig working conditions are riddled with issues {that vary across platforms, although ``even workers on the same platform can experience \dots differences'' \cite{supporting}}. While some challenges (e.g., low and unfair pay {\cite{6B4U}}, long and irregular hours {\cite{lehdonvirta2018flexibility}}) can be observed through quantitative data, {many} other factors {that critically contribute to unpaid/invisible labor} --- e.g., {emotional stressors in care work \cite{ming2023go, supporting}, }discrimination {\cite{disability}}, compromised safety standards {\cite{taylor2023physical}} --- can only be {captured via} qualitative {forms of }data that descriptively report {the issue within its applicable contexts}. 
{While quantitative data help stakeholders directly measure effects of algorithmic management on outcomes such as hours of engagement and pay \cite{6B4U}, 
narrative accounts help generate \cite{supporting}, document and raise awareness around new, nuanced and contextual factors that contribute to invisible labor and hidden risks \cite{lived_quality}, especially given the rapidly-evolving nature of platform policies and algorithms \cite{pacify}. 
In particular, \citet{policy_probes} highlighted the potential of worker-centered tools to ``spotlight workers’ lived experiences'' and bring oversights in labor regulation ``to the attention of regulatory bodies''.} Thus, an effective data-sharing tool should provide avenues for both quantitative and qualitative data contribution. 

\subsubsection*{\tcbox[colframe=lightgray, colback=lightgray] {DR 1.2}: \textbf{Provide {Trust via} Privacy, Security, and Data Control}}
\label{DR1.2}
{Within online communities where identity disclosures are optional, establishing trust is well-known problem that remains prevalent \cite{cmc, dynamo}.} To provide trust and safety to users who contribute their personal work data, a data-sharing tool should be equipped with 
appropriate security precautions to protect their data and {policies, as well as} configurable privacy options. 

{In prior investigations, gig workers prioritizing \textbf{security} concerns cited fears of ``backlash, harming reputation, and legal consequence'' \cite{sousveillance} from platforms such as ''breaking platform terms of service'' or retaliation tactics like ``shadow bans'' \cite{uuapp}, while others worried about releasing locational data to ``past problematic clients'' \cite{supporting}. To minimize risks of security breaches such as these, some recommended techniques like ``anonymization, aggregation and perturbation of data'' \cite{uuapp}, in addition to ways of affording workers the ``ability to revoke data access'' \cite{supporting}. \citet{supporting} further found workers to generally prefer sharing ``aggregate data but not individual data with peers, primarily due to concerns related to competition''.
{Hence, all} quantitative data should be anonymized while qualitative data should have anonymity-preserving sharing options{, }and no worker accounts} should not have permissions to view identifiable personal work data of peers. {Additionally,} a data-sharing mechanism should guarantee workers sufficient choice over the granularity of detail in what data they upload, {length of data persistence,} who they share their data with, as well as an agency over whether they may contribute quantitative or qualitative data. 

{Around \textbf{privacy}, workers of related work found ``trust [to] go hand in hand with privacy policies'' \cite{supporting}, therefore a data-sharing tool} should remain transparent about how uploaded data get used by the system.
{We note that despite the close ties of privacy to trust, attempts at eliciting privacy requirements uncovered a paradox where although ``workers were aware of the risks of sharing data'' \cite{uuapp, privacy}, they ``were largely unconcerned with their likelihood'' \cite{supporting, uuapp}, suggesting that without a working prototype of a data-sharing system simulating the in-situ experience of contributing and uploading on a daily basis, ''it can be difficult to imagine and consider related concerns with data privacy'' when workers lack ``practical experience engaging with civic tech or data activism''. This further underscores the importance of transparently disclosing to workers the types of data collected and how it gets used by the system.}

\subsubsection*{\tcbox[colframe=lightgray, colback=lightgray]{DR 1.3}: \textbf{Support Heterogeneous Worker Goals and Workflows}}
\label{DR1.3}
Prior investigations found differences in workers' workflows and goals, {creating ``divergent preferences on how to best upload data''} \cite{supporting} {and ``no consistency on the types of data'' to upload \cite{uuapp}}. 
{Additionally, \citet{accountable} found that workers integrate ``multiple tracking tools''
for income tracking and planning in their work routine to ``learn what the job is like, determine if their jobs are worth continuing, know how much they’re earning, monitor productivity, and manage work/life balance''.}
{While the objective is not to encourage all workers to use every available feature, }the system should provide workers multiple methods of data upload, a variety of worker-centered features to support different incentives{, as well as incorporate and centralize financial-tracking features}, to accommodate a more diverse set of {financial} workflows and goals.
For instance, while some Uber drivers might be curious about their estimated earnings for particular Quests, others might simply want to track their earnings per trip {\cite{zhang2023stakeholder}} --- workers should have methods for keeping track of both units of work. 


\subsection{DR 2: Facilitate Worker Collaboration \& {Cross-Stakeholder} Resource Sharing}
\label{DR2}
{While} gig workers {already} engage with online forums \cite{atom} and self-tracking tools \cite{accountable} {to exchange experiential knowledge for furthering their understanding of platforms and their own work,} {we are unaware of existing }online space{(s)} {that are designed for workers across gig platforms and domains to} contribute to a shared data repository, or {that connect workers to existing resources}. Below are four key guidelines for creating digital environments for gig workers in a way that fosters collectivism and organizes resources {that are} of benefit and use to workers.

\paragraph{\tcbox[colframe=babyblue, colback=babyblue]{DR 2.1}: \textbf{Encourage {Contributions that Inform Key Labor} {Initiatives}}} 

\label{DR2.1}
{While prior works} \cite{supporting, codesign, zhang2023stakeholder} identified {shared}
 concerns {around gig work that both policy experts and workers considered priorities (e.g. equity, fair pay, safety) }data surrounding those topics are scarce to nonexistent, due to platforms' reluctance to share. 
 {To rectify this data deficit, \citet{supporting} recommended using qualitative data such as ``personal anecdotes'' to pinpoint drivers of discrimination, ``digestible breakdowns of costs and earnings'' to educate and bring awareness to workers (and the public at large) about whether they making above minimum wage, and ``communication channel between workers and policy experts'' to facilitate worker reports of power imbalances with clients via data like ``cancellations and safety reports'' \cite{uuapp}.}

\paragraph{\tcbox[colframe=babyblue, colback=babyblue]{DR 2.2}: \textbf{Connecting Workers to Resources {of Other Stakeholders}}}
\label{DR2.2} 
{As self-employed individuals, gig workers shoulder several resource accountabilities (e.g., financial, network), in the absence of organizational support \cite{accountable}.}
In discussions with policymakers and advocates, \citet{codesign} received many pointers from organizations and advocates for resources targeted to gig workers{, including ``employee assistance and job training programs''}. Unfortunately, there is currently no centralized space for disseminating such information. 
{Possibly driven by a fear of factors like competition and spam content, gig workers are disincentivized from constructing open, Wikipedia-like portals where they collectively gather and use ``data, insights and contextualize information'' around work conditions \cite{uuapp}.}
{A d}ata-sharing tool should serve as {a portal for} connect{ing} workers to such {known} resources.

\paragraph{\tcbox[colframe=babyblue, colback=babyblue]{DR 2.3}: \textbf{Multi-Domain Support {\& Worker-Accessible Tools}}} 
\label{DR2.3}
{As described in \ref{Related_Work_Challenges}, gig work} span a variety of work domains \cite{impact,ming2023go,kuhn2019expanding}, making it crucial for a data-sharing tool reach workers providing different services{, especially since ``The different types of tasks gig workers engaged in affected their preferences on what gets shared, how it is submitted, and how often it is to be uploaded'' \cite{supporting}}. 
{To accommodate the heterogeneous workflows, workers needs and data types involved with varying gig domains, 
data-sharing systems should offer options that give workers the agency customize sharing preferences --- e.g., formats of data to upload, and what devices to upload from.} {For instance, \citet{6B4U} pointed out how workers performing physical services like grocery shopping often ``do not own a desktop computer, so any solution had to be easily accessible from a mobile device'', but workers offering digital services (e.g., online freelancers) may prefer desktop solutions that embed into their existing workflows. Thus, a data-sharing portal that caters to both online and offline service providers should be accessible via both phones and laptops.}

\paragraph{\tcbox[colframe=babyblue, colback=babyblue]{DR 2.4}: \textbf{Empower Collectivism {\& Cross-Stakeholder Communication}}}
\label{DR2.4}{While the value and necessity of achieving ``effective representation and collective bargaining for workers in the gig economy'' is widely recognized in research \cite{calacci2023access, workshop, reinvent, rights}, }the online and individual nature of work isolates workers from peers \cite{atom, commodified, alienated}{, making gig work collectivism the `holy grail' of the community}. 
In order to truly connect workers in a network {that benefits themselves instead of platforms \cite{commodified}}, the tool should allow for communication between gig workers, including those across platforms.
{Additionally, t}he system should also open up collaboration to {higher-power} stakeholders such as policymakers and advocacy groups {to ``find ways of maximizing their ability to support gig workers'' \cite{codesign}}. 

\section{Gig2Gether}\label{gig2gether}
{Based on an iterative design process, we developed Gig2Gether:} a worker-centered data-sharing tool with capabilities for uploading work data, viewing personal and collective work trends, sharing stories about work, as well as planning work and taxes. 
Built as a web app, Gig2Gether accommodates workers operating from various devices (laptop, mobile, \& any device with web-browsing capabilities). The app consists of a frontend built with SvelteKit and backend (database, storage and analytics) supported by Firebase.

\begin{figure}[h!]
    \centering
    \includegraphics[width=\linewidth]{Chapters/images/overview.png}
    \caption{Screenshots of Main Gig2Gether Features: Story Sharing and Feed (a), Income and Expense Uploads (d), Personal Trends (b), Collective Insights (e), Work Planner (c) and Tax Prep Resources (f)}
    \label{overview}
\end{figure}

Users can leverage the system to plan for, record and reflect on work at various stages of a job.
Before a gig, workers can use the planner to predict future earnings and set work goals. After finishing a task, {workers can store and share its} associated earnings, expenses and stories.
After uploading data {for} the recently completed task, workers can view and reflect on personal work trends, or use collective insights to grasp macro-level statistics about comparable or contrasting worker populations. 
Between gigs, workers can leverage  the 1) \textit{story feed} to learn about strategies or recent work conditions reported by peers, 2) \textit{tax page} to peruse resources that support fulfillment of tax obligations or 3) \textit{profile page} to reflect on their history with the platform or record repeatedly incurring expenses. 

\subsection{Exchanging Stories: Qualitative Data Sharing}
One of the intentions of Gig2Gether is to maintain a community for gig workers to share their own experiences with peers as well as policymakers and advocates, so as to help alleviate social isolation. To fulfill this objective, the Stories panel allows users to read and post about strategies and issues related to their everyday work. When sharing stories, workers are required to choose related tag(s), which are currently prepopulated with themes identified from Section \ref{Related_Work_Design} \nameref{Related_Work_Design}.


\paragraph{Share Story} \label{share_story}
Each story must 1) be shared as a strategy or issue, 2) be associated with at least one tag, 3) contain story content via a title or textual description, and 4) have a selection of desired viewing audience -- this can include other worker users of the system, policymakers, advocates, or be entirely private (i.e. visible only to themselves). Optionally, workers can include an image or video to provide additional context. See the share story page on the left of Figure \ref{overview}(a).

\paragraph{Story Feed} The story feed provides a place for workers to exchange stories with peers on Gig2Gether. 
At registration time, users are advised to choose a username that will be viewable to other users of the system, and each post is associated with the user only through the username. Posts can be filtered by the story type (Issue or Strategy), as well as by work platform (currently Uber, Rover or Upwork). Gig2Gether allows for cross-{platform} user interaction -- users can currently view and ``like'' posts via thumbs-up buttons. Commenting is currently unsupported, in the absence of an established moderation structure.
The feed is chronologically ordered -- most recent posts appear first; an example can be found via the right side of Figure \ref{overview}(a).
% \vspace{\baselineskip}

The story feature serves to meet \reftwo{DR2.4} since users can view data and strategies with other workers, support others' stories, as well as share strategies and insights gathered surrounding platform policies and functioning. 
By ensuring that workers have agency to configure desired viewing audiences of each shared post, and keeping users associated to their stories with only usernames, the stories feature also aligns with \refone{DR1.2}. Tags encourage the sharing stories related to initiatives of interest to policymakers, in observance of \reftwo{DR2.1}.

\subsection{Upload of Gig-related Earning \& Expense Data}
\label{upload}
One key feature of Gig2Gether is to help workers keep track of data surrounding their gigs so they can remain financially accountable. Below, we outline how workers of the three domains/platforms can upload income and expense entries. 

\paragraph{Income} 
For Rover and Upwork users, Gig2Gether currently only supports manual data entry. 
In the income form, a worker can upload information pertaining to time spent, earnings (including the platform cut and tips), as well as information specific to job types, such as time spent travelling to house sits (Rover) and experience levels (Upwork). 
An example for the Rover manual upload is shown at \ref{overview}(d).

\vspace{\baselineskip}

Uber users can manually upload data about Trips or upload CSVs that contain platform-collected data about their trips.
The \textit{Trip entry form} gathers information on the time spent, income, distance travelled, Uber fees, surge fees, as well as other specific items detailed in a Trip receipt. 
Finally, Uber allows drivers to download CSV files containing information on lifetime trips, payments and app analytics. Workers have a space to keep track of such information with Gig2Gether, offering a more expedited way of seeing personal work trends.


\paragraph{Expenses} Workers can manually input details about
incurred expenses related to gigs. To add an entry, users must enter the date and cost amount, while fields such as expense type, description and a photo uploads are optional for their own notetaking. Refer to \ref{overview}(d) for the expense upload page for Rover workers.

\vspace{\baselineskip}

In response to \refone{DR1.2}, income and expense uploads require only a small set of information: date, length and type of work, as well as income amount for income entries while expense entries only require data and amount of expense.
This way, workers retain agency over to choose the fields to share or track about income and expense entries. To further address the data control requirement, Gig2Gether provides manage data pages for users to view, modify and delete and story, expense, and income uploads at any point. 
In the income uploads, drivers have options to submit data manually or streamline the process by uploading their CSV's, in adherence to \refone{DR1.3}. Finally, the custom form fields of expense and income entries for each platform complies to multi-domain support outlined in \reftwo{DR2.3}.
    
\subsection{Viewing Work Trends}
To further educate workers about their own work, as well as insights surrounding other gig workers, we created two pages for workers to view both personal and collective trends, outlined below.

\paragraph{Personal Trends} 
To stay informed about earning patterns and work hours, workers can overview earnings, expense, hourly earning rates and hours worked in the ``My Trends'' page. Based on income and expense entries that users uploaded (process described in Section \ref{upload}), workers can view hourly and weekly earning trends, daily earnings by month, as well as summary statistics such as hourly pay and worked hours. The design of the hourly and calendar data visualizations in ``My Trends'' were informed by the personal data probes (in particular the Hourly and Calendar probes) from \citet{zhang2023stakeholder}. The Personal Trends page is displayed in \ref{overview}(b).

\paragraph{Collective Trends} In addition to personal metrics, workers can also view aggregate information about other Gig2Gether users via the ``Collective Insights'' page. At the time the study was conducted, this page is populated only with mock data rather than real data that workers inputted to protect the privacy of our small pool of test users. However, the page does include charts and options for dimensions of input (hourly income rate, tipping rate, and ratings) as well as demographic information to breakdown each dimension by (age, gender, ethnicity, income, education, tenure, and part/full-time work). Users can additionally compare their own data point against any breakdowns displayed. Refer to \ref{overview}(e) to view the Collective Insights page.

Both personal and collective trends map directly to the \refone{DR1.1}, and once collective trends is populated with real user data, all inputs will be anonymized to protect workers' privacy (\refone{DR1.2}).


\subsection{Planner}

\paragraph{Work Planning} Currently, Gig2Gether offers a prototype of a work planning feature that would inform its about predicted future earnings based on planned hours of work that users input and historical data. Currently, the Planner takes in a range of the days a user plans to work in, as well as hours they plan to work on those days, and displays a simulated summary and breakdown of what predicted earnings might look like. The current implementation of the Work Planner is displayed at \ref{overview}(c).
In the future, the Planner would populate the earning projections using users' historical data and work trends or patterns. 
Implementation of the Planner was guided by \refone{DR1.1} to help workers gain personal statistics, since predicted data is directly based on the user's history of uploaded information. The Planner is also based on the design, inputs, and outputs of the Planner data probe from \cite{zhang2023stakeholder}.

\subsection{Additional Features}

\paragraph{Tax Preparation} In adherence of \reftwo{DR2.2},
the tax page features resources for part-time and full time workers, guides from platforms, as well as general information about filing. It tracks the next tax day for eligible workers, in addition to providing information, resources, and tax preparation tips. To view the Tax Resources page, refer to \ref{overview}(f).

\paragraph{Multi-Domain Support}
Story sharing allows for cross-domain communication between users, following \reftwo{DR2.4} to inspire a gig worker collective. Additionally, workers have access to a variety of tax resources for all gig work domains. This helps workers who work multiple types of jobs to reference domain-specific tax resources for all three gig platforms (supported by Gig2Gether).

\section{Field Evaluation Methods}\label{field}
To assess the practical application of Gig2Gether in the daily working lives of various gig workers and how it can assist them in gathering evidence of issues to share with policymakers, we conducted a field study with 14 gig workers across the three domains. Workers were asked to use the system for 7 consecutive days, in addition to 1-hour onboarding and exit interviews.

\subsection{Recruitment}
We recruited gig workers through various channels, including r/Upwork, r/RoverPetSitting, and r/Uber subreddits. In addition to Reddit, we posted on city-specific Nextdoor and Craigslist, reached out to participants from prior studies and handed out flyers to Uber drivers in-person at airports. 
Interested individuals were required to complete pre-screening surveys to ensure eligibility and diversity in work types, locations, and experience levels. 
Selected participants then completed a consent form and a pre-study questionnaire to gather demographic information.
In total, we recruited 16 gig workers from different platforms (8 Uber drivers, 5 Rover petsitters, and 2 Upwork freelancers) with varied experience levels, as shown in Table \ref{participants}. Onboarding sessions and exit interviews were conducted via Zoom. Participants received up to \$200 USD as compensation, which included \$30 for onboarding, \$140 for the field study (\$15 per day for 7 days plus \$35 for optional tasks), and \$30 for the exit interview.

\begin{table}[]
\centering
\resizebox{1.2\textwidth}{!} {
\begin{tabular}{|l|l|l|l|l|l|p{1.5cm}|p{1.5cm}|}
\hline
\textbf{ID} & \textbf{Age} & \textbf{Gender} & \textbf{Ethnicity} & \textbf{Tenure} & \textbf{Education} & \textbf{Household income} & \textbf{Gig Work Status} \\ \hline
Driver-1 & 45-54 & Male & White & 2-5 years & High school/equivalent & \$25-50k & Full-Time \\ \hline
Driver-2 & 45-54 & Male & White & 0.5-1 year & Bachelor's & \textgreater{}\$150k & Part-Time \\ \hline
Driver-3 & 45-54 & Male & White & 1-2 years & Some college, no degree & \$50-75k & Part-Time \\ \hline
Driver-4* & 45-54 & Male & White & \textgreater 5 years & Some college, no degree & \$25-50k & Full-Time \\ \hline
Driver-5\textsuperscript{+} & 35-44 & Male & Asian & 2-5 years & Professional degree & \textgreater{}\$150k & Part-Time \\ \hline
Driver-6 & 45-54 & Male & Asian & >10 years & Some college, no degree & \$25-50k & Part-Time  \\ \hline
Driver-7 & 25-34 & Male & Hispanic/Latino & 2-5 years & Bachelor's & \$50-75k & Part-Time \\ \hline
Driver-8 & 35-44 & Male & Asian & >5 years & High school/equivalent & \$25-50k & \text{Full-Time} \\ \hline
Driver-9 & 35-44 & Male & White & >5 years & Bachelor's & \$75-100k & Part-Time \\ \hline
Freelancer-1 & 45-54 & Female & White & \textless{}0.5 years & Associate's & \$25-50k & Part-Time \\ \hline
Freelancer-2 & 25-34 & Female & White & \textgreater 5 years & Professional degree & \$100 - 150k & Part-Time \\ \hline
Petsitter-1 & 35-44 & Female & White & \textgreater 5 years & Some college, no degree & \textless \$25k & Part-Time \\ \hline
Petsitter-2 & 18-24 & Female & White & 0.5-1 year & High school/equivalent & \textless \$25k & Part-Time \\ \hline
Petsitter-3 & 25-34 & Female & White & 2-5 years & High school/equivalent & \textless \$25k & Full-Time \\ \hline
Petsitter-4 & 35-44 & Female & White & >10 years & Bachelor's & \$100 - 150k & Part-Time \\ \hline
Petsitter-5 & 25-34 & Male & White & 0.5-1 year & Master's & \$100 - 150k & Part-Time \\ \hline
\end{tabular}
}
\caption{Participant Demographics. \\
Our driver, petsitter, and freelancer participants engage with Uber, Rover and Upwork, respectively. \\
* Driver-4 dropped out after onboarding due to concerns that his participation would violate Uber policies. \\
\textsuperscript{+} Driver-5 dropped out after onboarding due to personal reasons, preventing him from actively uploading data.}
    \label{participants}
    
\end{table}

\subsection{Onboarding Interviews}
The field study commenced with a one-hour, one-on-one onboarding session to introduce participants to the study. At the beginning of each session, we guided participants to complete an income form for one of their recently completed tasks (e.g., an Uber trip, a Rover Task, or an Upwork job) while they screenshared. For the remainder of the session, we introduced the rest of the features of Gig2Gether. Participant's screenshares and real-time interaction allowed for immediate feedback and clarification. 
At the end of each session, we detailed the study's minimum requirements and optional tasks -- a copy of consent form, which includes the payment structure, was also sent to each participant via email. The daily task requirement rewards participants \$15 a day for completing one of:
\begin{enumerate}
    \item Upload entries on \textbf{expenses} incurred (e.g., gas, pet supplies, office supplies) or \textbf{incomes} earned, which include
    \begin{enumerate}
        \item  Trip for Uber
        \item Income forms for Rover or Upwork
    \end{enumerate}
    \item Share a story
\end{enumerate}
To earn the bonus, participants were expected to complete the daily task each of the 7 consecutive days, in addition to completing the following actions at least once: 1) Plan upcoming work, 2) View personal trends, and 3) Like another participant's story. 
Optionally, Uber drivers received the secondary option of earning the bonus by uploading a CSV of historical trips in lieu of the three actions stated above. 


\subsection{Exit Interviews} 
We conducted one-hour semi-structured exit interviews with each participant. Questions of the protocol focused on the key features of Gig2Gether—such as data uploading, trend analysis, storytelling, and the planning tool—as well as participants' overall experiences. Additionally, we tailored questions to the records of participants' daily interactions with Gig2Gether, including stories shared and uploaded income/expense entries. 

\subsection{Analysis Method}
To investigate workers' interactions with our system, we took a mixed-methods approach to 1) aggregate quantitative statistics about usage such as counts of stories/uploads shared and 2) qualitatively examine onboarding and exit interviews. 
For the quantitative data, usage reports were fetched and aggregated directly from the system backend, after which minimal calculations such as averages were performed.
For the Zoom-recorded interview transcripts, three researchers conducted open coding to to identify concepts, themes, and events. 

\section{Findings}
Below we report on our study findings, broken down by themes: first we describe {the role that Gig2Gether played in} participants' workflows as well as {(current and imagined future)} use cases regarding the tool; next we give an account of workers' stance on the system as a means to share data with policymakers and the types of information they prioritized to share; {finally, we present new considerations for a worker data-sharing system as surfaced from participants' use of Gig2Gether during the field study}.

%{finally, we present new considerations for a worker data-sharing system as surfaced from participants' use of Gig2Gether during the field study.}
\subsection{Worker Data-sharing in Practice{: Exchanging Support \& Insights While Managing Individual Finances}}
% More tenured workers (D6) saw more utility in the quantitative/financial tracking portions of the app while newer gig workers engaged more with the Story Feed (P2).
% <-- try to verify this with data
During onboarding, participants expressed initial reactions to how they envisioned using features of Gig2Gether. 
{In} exit interview{s after the 7-day field study}, workers {shared further }details about {existing and desired use cases.}
In the following, we present findings about how participants used features of Gig2Gether using contextual details they revealed during interviews and usage metrics gathered from the system.

% Since the \textit{Collective Insights} and \textit{Planner} primarily served to illustrate the potentials of data-backed forecasts and breakdowns, participants shared many ideas on imagined use cases and proposed features -- we detail these suggestions in \ref{Findings_Improvement}.

% Overall, provide overview that users were able to able to use each features, then dive deeper into each feature
% descriptive table of how much users used each feature

%I am ignoring all this text below about excitement and insights from the onboarding session.
%what people were using, web analytics 
% \subsection{Safety: Physical and Digital}
%can include table about web-tool analytics -- time spent on pages, which pages most interesting
%report mostly feedback and modifications suggested/reQuested from the exit interviews? 
%how people used !) stories (qual), 2) data uploads, and 3) planner or personal analytics dashboard? 
%%Focus on Exit Interview Notes
%\subsection{Service-specific Feedback}
%Angie ^^ I suggest depending on exit interviews, we remove this and include as subsubsections the breakdown of work types ?

%\subsection{Tool Usage}
%connect back to \cite{lived}, i.e. current patterns of use and it was/could be enmeshed in current workflows/lives
\subsubsection{{Solidarity \& Collectivism via Experience \&} Data Exchange} 

Many workers described \textit{Stories} as a unique feature distinguishing Gig2Gether from other data-tracking or -sharing apps they use. Though not everyone shared, many workers found it reassuring to read others' stories, since they get to \textbf{learn that they're not alone in experiencing hurdles} at work: ``I really like the fact that there's stories, and you can check out what everybody else is dealing with. So you feel like: Oh, I guess it's not just me that's feeling like they're \dots be[ing] cheated'' (Freelancer-1). Freelancer-2 shared the desire of wanting to connect with others, because ``You can really feel siloed as a gig worker sometimes, so it's cool to see other people's experiences''. When first reacting to the story feed during onboarding, Petsitter-5 immediately expressed resonance with a story: ``I have similar feedback ... I'll be adding a story shortly, because it's hard to get [jobs on Rover] versus \dots WAG. Yeah, definitely want to talk about this.''
A few workers specifically pointed out the content and attitudinal contrast of Gig2Gether's story feed with other gig work forums: ``the subreddit is really just a lot of sharing \dots but not necessarily useful [content] \dots but [Gig2Gether] offers tools'' (Freelancer-2). One participant even expressed considered sharing stories to initiate collective action against Uber:
\begin{table}[h!]
\centering
\resizebox{\textwidth}{!} {
\begin{tabular}{cc|ccc|c}
\cline{3-5}
 &  & \multicolumn{3}{c|}{\textbf{Authors's Work Contexts}} &  \\ \cline{3-6} 
 &  & \multicolumn{1}{c|}{\textit{Driver Stories}} & \multicolumn{1}{c|}{\textit{Petsitter Stories}} & \textit{Freelancer Stories} & Total \\ \cline{2-6} 
\multicolumn{1}{c|}{} & Total Authored & \multicolumn{1}{c|}{15} & \multicolumn{1}{c|}{11} & 1 & 27 \\ \cline{2-6} 
\multicolumn{1}{c|}{} & Mean Stories / User & \multicolumn{1}{c|}{2.143} & \multicolumn{1}{c|}{2.2} & 0.5 & N/A \\ \hline
\multicolumn{1}{|c|}{} & \textit{From Drivers} & \multicolumn{1}{c|}{13} & \multicolumn{1}{c|}{\textbf{10}} & 0 & 23 \\ \cline{2-6} 
\multicolumn{1}{|c|}{} & \textit{From Petsitters} & \multicolumn{1}{c|}{\textbf{10}} & \multicolumn{1}{c|}{10} & 0 & 20 \\ \cline{2-6} 
\multicolumn{1}{|c|}{\multirow{-3}{*}{\textbf{\# Likes}}} & \textit{From Freelancers} & \multicolumn{1}{c|}{\textbf{1}} & \multicolumn{1}{c|}{0} & 0 & 1 \\ \hline
\multicolumn{1}{|c|}{} & \textit{Workers Only} & \multicolumn{1}{c|}{2} & \multicolumn{1}{c|}{1} & 0 & 3 \\ \cline{2-6} 
\multicolumn{1}{|c|}{} & \textit{Policymakers Only} & \multicolumn{1}{c|}{0} & \multicolumn{1}{c|}{1} & 0 & 1 \\ \cline{2-6} 
\multicolumn{1}{|c|}{} & \textit{Workers + Policymakers} & \multicolumn{1}{c|}{1} & \multicolumn{1}{c|}{0} & 0 & 1 \\ \cline{2-6} 
\multicolumn{1}{|c|}{\multirow{-4}{*}{\textbf{Share to}}} & \textit{Workers + Policymakers + Advocates} & \multicolumn{1}{c|}{12} & \multicolumn{1}{c|}{9} & 1 & 22 \\ \hline
\multicolumn{1}{|c|}{} & \textit{Strategies} & \multicolumn{1}{c|}{10} & \multicolumn{1}{c|}{8} & 0 & 18 \\ \cline{2-6} 
\multicolumn{1}{|c|}{\multirow{-2}{*}{\textbf{Story Type}}} & \textit{Issues} & \multicolumn{1}{c|}{5} & \multicolumn{1}{c|}{3} & 1 & 9 \\ \hline
\end{tabular}
}
\caption{{Story statistics across platforms. 
\textit{Note how workers of all platforms expressed interests (through likes) for a comparable number of stories in their domains as in others --- e.g., drivers liked 10 stories from petsitters, in addition to 13 stories from other drivers}}}
\label{story_stats}
\end{table}
\begin{quote}
There's a lot of things that I would like to share, but most of them are political. So like: we should all get together, fight back against Uber \dots [but] I didn't know how political I could be [on the Story feed].
\end{quote}

Several participants shared a displayed level of \textbf{interest in other platforms} supported by {Gig2Gether}---either they had prior interest or developed interest for how to start work on another platform after reading others' stories. In both cases, workers found value in reading about others' experiential strategies and issues. {This interest in other platforms' users' stories was reflected in usage metrics (see Table \ref{story_stats}), which show how platforms' workers expressed support (via likes) for a comparable number of stories in their own domain as from other domains (likes from other platforms are bolded).} Petsitter-3 is now considering both Uber and Upwork as extra sources of income: ``I did [like] one [story] from Upwork because I was actually looking to work there at some point \dots I saw a lot of pointers that people gave for Upwork, and I was like, `You know what? I'm gonna definitely keep that in mind.' '' 


\paragraph{\textbf{{Collaborative Examinations of Algorithmic Speculations \& Rate Standardization}}} Although the \textit{Collective Insights} page was not yet populated with real user data, it sparked ideas and hope in participants for what could be revealed with aggregated data. For instance, Driver-2 expressed excitement about the potential of \textbf{answering popular speculations} about effects of having a Tesla on Uber earnings: ``[on] the Reddit Forums for Uber drivers, people are always asking `if I buy a Tesla (or if I get an XL) what should I expect as far as [how much] my tipping [were] to increase, or hourly income to increase?' So this is actually pretty cool''. In addition to large differences such as car model, Driver-7 wondered whether small gestures such as amenities can affect earnings: ``car model \dots [and] the type of amenities that the driver offers''. On Rover, Petsitter-2 also wished to confirm her own observation-based hypothesis that ``vets have a lot more repeating customers \dots they also tend to be the more expensive ones''. Beyond helping workers decide the type of services to offer, participants also saw collective insights as a tool to help them \textbf{set rates of charge for services}. Petsitter-4 expressed how
\begin{quote}
    I would love to see [earning statistics] broken out by urban, suburban, rural \dots [because] that's the biggest difference in how sitters operate \dots it's a entirely different game. Right now I'm urban, I have a radius of two miles and I walk to all of my bookings, whereas a rural sitter might have a radius of like 10 miles, where they'd have substantial costs in terms of travel time and driving \dots [So urbanization would impact how] I set my pay rate. 
\end{quote}
Driver-3 similarly wondered about fare price difference across geographic regions: ``The only [additional breakdown I'd want] \dots would be your region \dots I noticed different fare prices of getting out of the city''. 


In online freelancing, platforms offer a wide variety of job categories, thus Freelancer-2 desired to find out about differences between and intersections of categories: ``I work in healthcare but a lot of the work I do on Upwork is writing, it would be interesting to see \dots [the breakdown or] an overlap of both categories.'' Freelancer-2 {also} offered the idea of breakdowns by disabilities: ``physical and mental disability, might also be a good differentiator there''.


\noindent

\subsubsection{ {Financial} Tracking{: Self-Logging $\rightarrow$ Reflections \& Planning}}
\paragraph{\textbf{Streamlined Financial Tracking}} Participants described Gig2Gether as straightforward (``simple'' and ``easy'') to use when manually entering information. 
While various third party apps emerged over the years to help workers track earnings, expenses, and tax obligations --- as noted in \S\ref{Related_Work} and by workers such as Driver-1 -- some participants (particularly from {non-driving domains}) preferred Gig2Gether over such apps for its simplicity: ``I like this way better, because this is for gig workers, and the other is more of the financial crap that I don't like having to deal with, but I do [have to]'' (Freelancer-1). Petsitter-5 also enjoyed the simplified experience of viewing his financial data: ``this is better than [Rover. There] it's just too complicated. And I love seeing how data is simplified [here]''.
Although rideshare driving often accumulates a larger volume ``gigs'' in a day than petsitting or freelancing, D3 (who does not currently use tracking tools) expressed a similar desire to using Gig2Gether: ``I don't always remember everything, but I can keep it all just between the Uber and my head. But I would like to use a simplified [tracker]---another platform like you guys are presenting now.'' 
\begin{table}[h]
\centering
\resizebox{\textwidth}{!} {
\begin{tabular}{ccccccc}
\textbf{} &
  \textbf{\begin{tabular}[c]{@{}c@{}}\# Shared \\ Stories\end{tabular}} &
  \textbf{\begin{tabular}[c]{@{}c@{}}\# Total Words \\ in Stories\end{tabular}} &
  \textbf{\begin{tabular}[c]{@{}c@{}}\# Liked \\ Stories\end{tabular}} &
  \textbf{\begin{tabular}[c]{@{}c@{}}\# Income \\ Uploads\end{tabular}} &
  \textbf{\begin{tabular}[c]{@{}c@{}}\# Expense \\ Uploads\end{tabular}} &
  \textbf{\begin{tabular}[c]{@{}c@{}}\# Trends \\ Visits\end{tabular}} \\ \cline{2-7} 
\multicolumn{1}{c|}{\textit{Average}} & 1.93 & 231  & 3  & 8.57 & 1.42 & 4.5 \\
\multicolumn{1}{c|}{\textit{Median}}  & 1    & 108  & 3  & 6.5  & 1    & 4.5 \\
\multicolumn{1}{c|}{\textit{Max}}              & 5    & 1493 & 11 & 41   & 7    & 9   \\
\multicolumn{1}{c|}{Total}            & 27   & 3235 & 42 & 120  & 20   & 63 
\end{tabular}
}
\caption{Descriptive Statistics on Stories, Uploads and Trends}
\label{summary_stats}
\end{table}
\paragraph{\textbf{{Integrating Financial Reflections into Gig Workflows}}}
Beyond the initial income entries uploaded during onboarding, all participants entered additional income entries, and 9 of 16 uploaded expense entries {-- with one participant entering 7 expenses (see Table \ref{summary_stats})}. Based on these uploads, participants reflected on the personal earnings presented by the Trends page. Driver-2 appreciated having the ability to review his work data: ``I really liked the Trends section. Uber doesn’t give trends, just reports. And the Trends helped to look back and choose the weekends and decide what times are best to work.'' During onboarding, (part-time) Petsitter-3 expressed similar excitement about being able to compare earnings across time: ``Rover doesn't have something like this where you can \dots compare this year to last year.'' 
Driver-6 was excited by the ability to view his weekly data{, and} wanted to use the Trends page to show his friends proof of earnings having gone down, e.g., from working the same amount of time, year over year:
\begin{quote}
    If I have this app, then I can show them the facts \dots this is this [amount] before, and this is now \dots it can actually affect if I still want to do this Uber thing, or I can tell my friends not to do it anymore. \dots Because this is data, this is like facts 
\end{quote}
Driver-3 went a bit further to imagine how historical data can help him plan for breaks:
\begin{quote}
So you can cut out with Uber, cut out some downtime with the slow hours \dots actually have a break and not worry about missing anything. \dots [I could] look back to last year and say \dots September 1 was busy, and it slowed down at 10 o'clock \dots So you don't have to waste your time staying on the app    
\end{quote}

Driver-3 liked recording and seeing expenses displayed back, explaining how Trends could help him ``streamline my expenses a little better \dots when I use plastic [cards to pay], I don't pay attention as much as you do when you're handing cash over \dots [but] with having your site up, I could just go back and refer to everything and say, `Hey, maybe slow down on this' \dots when I'm going through expenses.''
In a similar manner, Driver-6 enjoyed entering his information at the end of a day that he had driven. In his 10 years of driving, he had never been compelled to try recommended apps from fellow drivers, but found it easy to use Gig2Gether to enter information and subsequently view Trends.

Freelancer-1 also shared the enthusiasm for potentially streamlining work processes such as tax filings: ``This is super helpful. If [only] I would have had this when I was helping my husband with starting up his stuff \dots the whole tax thing was a nightmare for me''.
Petsitter-1 described her affinity for the tailored aspects of the tool by contrasting against how most similar apps frame gig workers as independent contractors, which misaligns with the reality of their work and earnings: ``A lot of that stuff is like: `Get \textit{blah, blah blah} for your small business.' [But] I'm not a small business owner''. 
Several participants talked about creating reminders to remember to upload their data, such as Freelancer-2 ``a reminder in my calendar just to make sure I wouldn't forget'' and Driver-3, who had to ``set a reminder to make sure I did [uploads for daily tasks]''.
However, uploads can became a part of normal work routine---Petsitter-3 added it to the ``housekeeping things I needed to do, and it seemed to flow pretty naturally in with those reminders.'' In the same manner, Petsitter-4 also mentioned push notifications would help but were not necessary because ``anytime you start something new, it's not habit yet \dots just have to get used to it''.

\paragraph{\textbf{{Planning, Keeping up with \& Achieving Earning Goals}}} Although the Gig2Gether \textit{Worker Planner} was only populated with mock data, participants were eager to incorporate it into their workflow{, and} resonated with the the planner's goal of helping them structure schedules for days and review earnings goals. Freelancer participants foresee themselves using the \textit{Planner} to track true hourly wages after expenses: Freelancer-2 would use it to``keep things straight \dots [So I can compare:] I work this much this week. This is how much I uploaded [in earnings]'', while Freelancer-1 would use it to check ``how much I'm working, whether my expenses offset with the money I'm making. And see if I need to work more''. Driver-3 envisioned using the \textit{Planner} to help remember and plan around upcoming reservations, which can go as far out as 30 days:
``I would definitely use it a lot, because of the reservation rides \dots tonight I have a reservation ride for [which] I can't remember [the exact time]''. Petsitters held mixed opinions about the \textit{Planner}, partially due to how Rover already provides a calender for bookings - we outline some suggestions they made in \ref{planner_improvements}.

\subsection{Data Disclosures to Policymakers, Peers \& Advocates}\label{Findings_Sharing}
{Workers also envisioned several potential ways of impacting policy or mobilizing collective action for several initiatives, which we describe below}.

\subsubsection{Openness to Data Sharing with Policymakers and Advocates}
Workers expressed strong support for Gig2Gether's mission of shedding light on their working conditions to policymakers: ``This is a tool that's designed to bring exposure to policy makers \dots \textbf{to open the door between drivers and politicians} \dots now that could interest a lot of drivers right there.'' (Driver-1). 
{Through shared stories, we note workers were willing to share their qualitative data with policymakers in 23 of 27 cases (Table \ref{story_stats}). }
Beyond a willingness to share data with policymakers, workers also shared preferences for prioritized issues such as safety and wage concerns. With Gig2Gether, they hoped advocates and policymakers will ``get out the realistic facts of the jobs'' (Petsitter-1). Even when they were not sure how a story or metric could relate to policymaking, participants exhibited a general desire for their data to simply raise awareness about their work conditions: ``You could share that [data] because I don't think anything would hurt anyone. If anything, it'll maybe open some eyes up.'' (Driver-3). 

\subsubsection{{{Story Feed: A More Reputable Source for Informing Policymaking}}}
When comparing the \textit{Story} feed to other online forums they engaged with, participants considered Gig2Gether as a more credible source, which may make (1) policy stakeholders take it more seriously, and (2) workers more comfortable interacting with other workers. Petsitter-4 explained her rationale for increased trust in Gig2Gether: 
\begin{quote}
I would feel a little bit more comfortable that I was getting information from like verified sitters \dots
it would be weird for someone to sign up for an app to track their earnings, and then shit post in the community section of it \dots It would be a community that would be obviously a little bit more verified, and then a little bit more serious \dots [with members who are] committed to gig work, to a point where you're going to the trouble of tracking your earnings/expenses''.     
\end{quote}
Sharing this sentiment of increased reputation/trust in Gig2Gether, Freelancer-2 postulated on its effects on policymaker perceptions: ``I feel like \dots they might disregard what they saw on Reddit \dots [as] people venting online, people being bitter\dots but if it was coming from a more reputable forum \dots They might take it to heart a little bit more.'' Driver-2 also compared it to Reddit, saying Gig2Gether represents a place with less trolls, where workers are ``planning for more success''.
Meanwhile, Petsitter-1 shared her thoughts about the role of advocates in disseminating information about the platform: ``[Gig2Gether and its insights] is the kind of thing that I think would be better spread through advocacy groups than through individual word of mouth ''
\begin{table}[h!]
\centering
\resizebox{\textwidth}{!} {
\begin{tabular}{c|ccc|c|cc|c|ccc}
\multicolumn{1}{l|}{} & \multicolumn{3}{c|}{\textbf{Usage in Authored Stories}} & \multicolumn{1}{l|}{\multirow{2}{*}{\begin{tabular}[c]{@{}l@{}}Total \\ Usage\end{tabular}}} & \multicolumn{2}{c|}{\textbf{Story Type}} & \multicolumn{1}{l|}{\multirow{2}{*}{\begin{tabular}[c]{@{}l@{}}Total \\ Liked\end{tabular}}} & \multicolumn{3}{c|}{\textbf{Liked Stories}} \\ \cline{2-4} \cline{6-7} \cline{9-11} 
\multicolumn{1}{l|}{} & \textit{Driver} & \textit{Petsitter} & \textit{Freelancer} & \multicolumn{1}{l|}{} & \textit{Strategy} & \textit{Issue} & \multicolumn{1}{l|}{} & \textit{Driver} & \textit{Petsitter} & \multicolumn{1}{c|}{\textit{Freelancer}} \\ \hline
\textit{\textbf{safety}} & \multicolumn{1}{c|}{5} & \multicolumn{1}{c|}{5} & 0 & 10 & \multicolumn{1}{c|}{7} & 3 & 19 & \multicolumn{1}{c|}{11} & \multicolumn{1}{c|}{8} & \multicolumn{1}{c|}{0} \\ \hline
\textit{\textbf{fair pay}} & \multicolumn{1}{c|}{5} & \multicolumn{1}{c|}{1} & 0 & 6 & \multicolumn{1}{c|}{4} & 2 & 5 & \multicolumn{1}{c|}{3} & \multicolumn{1}{c|}{2} & \multicolumn{1}{c|}{0} \\ \hline
\textit{\textbf{care-giving}} & \multicolumn{1}{c|}{1} & \multicolumn{1}{c|}{4} & 0 & 5 & \multicolumn{1}{c|}{3} & 2 & 9 & \multicolumn{1}{c|}{5} & \multicolumn{1}{c|}{4} & \multicolumn{1}{c|}{0} \\ \hline
\textit{\textbf{stress}} & \multicolumn{1}{c|}{1} & \multicolumn{1}{c|}{3} & 0 & 4 & \multicolumn{1}{c|}{2} & 2 & 6 & \multicolumn{1}{c|}{4} & \multicolumn{1}{c|}{2} & \multicolumn{1}{c|}{0} \\ \hline
\textit{\textbf{technology}} & \multicolumn{1}{c|}{3} & \multicolumn{1}{c|}{1} & 0 & 4 & \multicolumn{1}{c|}{4} & 0 & 10 & \multicolumn{1}{c|}{5} & \multicolumn{1}{c|}{4} & \multicolumn{1}{c|}{1} \\ \hline
\textit{\textbf{other}} & \multicolumn{1}{c|}{1} & \multicolumn{1}{c|}{2} & 1 & 3 & \multicolumn{1}{c|}{2} & 2 & 3 & \multicolumn{1}{c|}{1} & \multicolumn{1}{c|}{2} & \multicolumn{1}{c|}{0} \\ \hline
\textit{\textbf{ratings}} & \multicolumn{1}{c|}{1} & \multicolumn{1}{c|}{2} & 0 & 3 & \multicolumn{1}{c|}{1} & 2 & 6 & \multicolumn{1}{c|}{3} & \multicolumn{1}{c|}{3} & \multicolumn{1}{c|}{0} \\ \hline
\textit{\textbf{work time}} & \multicolumn{1}{c|}{0} & \multicolumn{1}{c|}{2} & 0 & 2 & \multicolumn{1}{c|}{2} & 0 & 3 & \multicolumn{1}{c|}{1} & \multicolumn{1}{c|}{2} & \multicolumn{1}{c|}{0} \\ \hline
\textit{\textbf{algorithms}} & \multicolumn{1}{c|}{0} & \multicolumn{1}{c|}{1} & 0 & 1 & \multicolumn{1}{c|}{0} & 1 & 2 & \multicolumn{1}{c|}{1} & \multicolumn{1}{c|}{1} & \multicolumn{1}{c|}{0} \\ \hline
\textit{\textbf{discrimination}} & \multicolumn{1}{c|}{1} & \multicolumn{1}{c|}{0} & 0 & 1 & \multicolumn{1}{c|}{1} & 0 & 1 & \multicolumn{1}{c|}{1} & \multicolumn{1}{c|}{0} & \multicolumn{1}{c|}{0} \\ \hline
\textbf{Total} & \multicolumn{1}{c|}{18} & \multicolumn{1}{c|}{21} & 1 & \textbf{39} & \multicolumn{1}{c|}{26} & 14 & \textbf{64} & \multicolumn{1}{c|}{35} & \multicolumn{1}{c|}{28} & 1
\end{tabular}
}
\caption{Tag Statistics Across Platforms}
\label{tags}
\end{table}

{Using} the \textit{Story} feed, workers shared more strategies than issues, with ``safety'' and ``{fair pay}'' emerging as the most used tags. 
For strategies, many workers talked about staying safe in the face of challenging client interactions (for Rover users, ``client'' can refer to the pet and/or its owner), such as Petsitter-1 when watching multiple dogs and Driver-1 when faced with trespassing customers. 
Experienced workers also shared strategies for improving earnings, such as methods for attracting repeat customers (Petsitter-2), testing platform features (Driver-2), recording unpaid work/time (Petsitter-4), or even a workaround for platform's evasions of small fees --- by tracking them and filing small claims lawsuits (Driver-1). 

In terms of issues, workers most commonly shared experiences of unsafe working conditions ---such as a driver writing about a stressful trip to drive an elderly man in distress to the ER. 
Participants also shared frustrations around understanding how algorithms assign work (Drivers-3,7) and concerns of power imbalances with clients (Petsitter-2, Freelancer-1).

\subsubsection{{Safety \& Wages}} 

{During exit interviews, we probed workers about how and which these shared concerns should be communicated to policymakers.} Below we detail examples of compromises to their safety or pay. 

\paragraph{\textbf{Understanding Worker Safety}}
When discussing safety concerns, participants referred to \textbf{physical safety} issues they face from riders (Uber) and pets and/or their owners (Rover), as well as \textbf{digital scams} (all platforms). We describe below the various physical dangers, since participants did not prioritize digital scams as a concern to share with policymakers.

Driver-1 described various factors that drivers might encounter ``incidents \dots like physical assault, being disorderly, and causing damage to the driver's vehicle (this happens pretty often), passengers getting arrested out of the back of your car'', which motivates him to use a channel such as Gig2Gether's \textit{Story} feed to funnel the information to policymakers, since it ``would be good to be able to report that somewhere centralized so that they can see there's a big safety issue.''
Beyond road conditions, safety risks can also encountered at strangers' homes ``you're going into somebody's house, it's a vulnerable position to provide work" (Petsitter-4). Furthermore, both participants pointed out how many of the risks imposed on workers are one-sided to protect consumer identify and safety: ``sitters are background checked, clients are not'' (Petsitter-4), and Driver-9 related being required to pass``a pretty rigorous background check \dots Both initially, and then it happens randomly. Usually only once a year, but it \dots has been more often''. Driver-1 described how prior to the \#WHATSMYNAME movement, ``the passenger would give their name to the drivers so that the driver knows that they have the right person'', but nowadays drivers have no method of verifying whether they have the correct person, causing breaches of safety because 
\begin{quote}
    You got young, beautiful women in their early twenties out there driving, and some big, burly dude opens the door [and asks] `What's my name?' Whatever name she puts out, he could say yes, [and] she could disappear from that point.
\end{quote}

\paragraph{\textbf{Understanding Unfair or Unpaid Wages}}
Participants of the three platforms described scenarios related to unfair or unpaid wages. Senior Drivers-1, -6, and -7 all lamented how Uber wages and incentives keep dropping over the years: ``when Uber first started, we were making like almost \$40 an hour. Now it keeps on going down \dots [on] the Quests right now, you just make make \$15 on 20 trips \dots they're getting so greedy'' (Driver-6) and ``Uber has gotten worse, and this might be my last summer [with them]''. For many of these rideshare drivers, gaining access to collective evidence is quintiessential for exposing the rapidly wage declines: ``The reason why \textbf{this data is important is because we want to expose literally what we're making. We want these policymakers to see this}''. Even more junior workers such as Driver-10 expressed desires to use the app to record subminimal wages: ``some states \dots looked at it and said, this [wage] is not fair. So I think that's probably where I would use the data that's within your app to basically show, `hey, here's what's really going on`. ''



\subsection{New Data, Metrics and Features}\label{Findings_Improvement}
{While research has explored workers' preferences for contributing data and how data can be used \cite{uuapp, supporting}, having workers test a prototype can allow them to surface important needs and opportunities that only arise from hands-on experience \cite{rogers2007s}. We found this to be the case where participants' use of Gig2Gether revealed important workflows to support, opportunities of insights to strengthen personal goals and collective action, as well as worker-to-worker interaction and anonymity preferences that would have otherwise remained unknown.} %insitu and testing of prototypes 

\subsubsection{{Insights About Essential Workflows to Support in Data-Sharing Systems}}

{As participants described their experiences using Gig2Gether to log their work, they highlighted additional important workflows that must supported for them to obtain the most useful insights about their work and earnings.}

\paragraph{\textbf{{Towards} Complete \& Automatic Data Uploads}} 
Several participants talked about taking gigs off-app (Petsitter-3, Freelancer-2) or {working} multiple apps (Drivers-1, 6, 7). For example, Driver-7 has shifted ``90\%'' of his work to Lyft so the Trends page would not reflect all his gig work earnings. He currently uses an Excel spreadsheet to manually input his weekly summaries from both platforms but wants an app that helps him track both.
{By describing their experiences uploading data and viewing their trends, participants highlighted the importance that} future versions of Gig2Gether {support workflows of multiple apps and off-app work so the} Trends page {allows them} a holistic {and meaning} view of net earnings and patterns. 

{Additionally,} related to data completeness, several Uber drivers described a need for automatic data entry support (Drivers-1,7, 9), similar to existing third party apps (most of which require paid subscriptions) that emulate actions on Uber/Lyft such as accepting or declining ride offers \footnote{Examples include Mystro and Para, both of which are paid apps}. Especially for full-time and long-tenured drivers, the volume of trips they accumulate can be substantial. Even though we offer a CSV upload option {for Uber drivers} to mitigate the process, drivers {describing the many trips they complete a day and the normalcy of switching between apps highlighted the importance of automatically gathering their work data to support complete data insights}. {When asked about concerns around data privacy if their accounts were linked, drivers did not have any and were supportive of a more automatic option.} {On the other hand, we anticipated that Petsitters or Freelancers would not require an option of bulk data upload given the nature of their work, and correspondingly, they did not share a need for automatic data entry.}


\paragraph{\textbf{A One-Stop Shop {For Understanding Profit \& Filing Taxes}}}
{Participants' descriptions of their experiences entering completed work and accumulated expenses also helped us recognize a possibly under-supported task in the ecosystem of gig worker tools: the tax-filing process.}
Driver-6 and Petsitter-1 {described desiring a "one-stop shop", which for them} translates into {one tool that lets them} pull all their data for purposes such as submitting to an accountant, IRS audit, tax filing, or if they're just curious:
``a one stop shop [so] that I can see my progress. I can see how much I'm making per hour. I can see my expenses. At the same time that I can show it to my accountant. Or if there's an audit from IRS, that I can show this.'' (Driver-6). Mileage, in particular, {was highlighted by both drivers and petsitters as an important metric for accurately calculating their expenses for filing taxes under the standard deduction:} ``I would like \dots [for this app to have] as opposed to them [other apps/forums] `one stop shop', if it had the mileage.'' (Petsitter-4).

{A couple participants described using a combination of apps to collect all the metrics related to understanding their work and filing their taxes.} Driver-7 uses {Stride to automatically track miles and Excel to manually log trip earnings.} 
{He expressed that Gig2Gether automatically} tracking miles would {complete the metrics} he needed {in a tool}---mileage, earnings and expenses altogether. 
{Driver-9 also shared using multiple apps for tracking mileage (Gridwise) and fuel expenses (Fuelly). He explained that Gig2Gether tracking miles} would further motivate him to share the app with friends ``I would recommend it, because it's more immersive than the other app that I use [especially if it can also be used] to track your mileage''.

{Not all participants desired this though.} Driver-8 warned us against trying to {expand Gig2Gether's features to fulfill} a ``one-stop shop'', {expressing worry around the} risks of chasing down an {endless} pool of desired features. Instead, he encouraged us to pursue Gig2Gether as a tool for connecting workers with policymakers and advocates as this was the unique feature he had not seen in past applications.
\paragraph{\textbf{Providing Context on the Planner}} \label{planner_improvements}
{While the \textit{Planner} was primarily presented in Gig2Gether as a predictive tool to project weekly earnings, participants offered different ideas for how they wanted to use it by describing their current work planning process.}
{For instance, though Uber's traditional model has been on-demand ride requests, they began letting riders schedule a ride request in advance---"Reservations". Driver-3 actively accepts these trips and wanted} to use the planner to keep him accountable for reservations. {Meanwhile} Petsitter-2 {wanted to use the \textit{Planner} to keep track of the different pets she's booked to care for}:
``Say I got Ice or I got Henny \dots I put their names all throughout the planner \dots Because sometimes I get them back to back and it'll be like: `Okay, wait, who's this one? ' '' Petsitter-3 also entertained ``the possibility to write in who I'm pet sitting for'' and further suggested the idea of ``being able to put in the address'' to each entry.

{Several drivers also highlighted their tendency to center their driving locations and hours based on events. Thus, they} pointed out the utility of incorporating regional ``events that would be in the city'', but not bigger ones because Uber already keeps track of those. Driver-7 {clarified how seeing events integrated into the \textit{Planner} would help smooth out his current workflow, as right now} he resorts to manually looking up such events himself, which can be time-consuming: ``I have to go online \dots [to look up when], Chicago Cubs play \dots write down on calendar by hand the right time \dots''.

% (which is otherwise pretty fixed right now); wanna compare projected earnings basd on his own data vs. others 
% Intergates with new features on apps as well now such as reservations for rideshare 
% michelle: rather than work planner, suggested tax planner (estimated taxes that she should pay). but imagined planner and individual trends like hourly/daily would be helpful for a multi-gig platform worker
% \textbf{\emph{Participants want to integrate goal-setting and accountability into the Work Planner to support their work profits and time spent working.}.} %i don't like the latter half "to support their work profits"
\subsubsection{{New Metrics That Can Strengthen Personal Goals and Collective Action}}
%\textcolor{red}{condense and shorten}
\paragraph{\textbf{Net Earnings Insights: Achieving Personal Goals and Empowering Collective Action}} %freelancer and driver

{Reviewing quantitative metrics on Gig2Gether, participants talked about how these can help support their personal goals as well as ideas they have for advancing collective action.}
Uber and Upwork participants yearned to view their net earnings (Freelancer-1, Driver-1), so they can plan for and achieve work goals. For instance, Freelancer-1 wanted to ``see how much I'm working with my expenses offset with how much money I'm making, and see if I need to work more''. Beyond personal earning goals, Driver-7 wanted to leverage trends from earnings data to show other drivers unfair or demanding working conditions imposed by platforms, explaining how workers often focus only on gross income without critically assessing their expenses. For instance, he'd want a way to show drivers whose net earnings are below minimum wage---e.g., 15 hours to make \$200. These statistics ``give them an insight of what's going on \dots [that] they're not making enough money'', so as to galvanize them to strike against rideshare companies, because ``in order to make a change, we [as drivers] have to get together'' in protest.
%In tandem with the excitement about comparing earnings across time (see \ref{time}), workers desired to further breakdowns earnings by time. Petsitter-3 wanted to view week-by-week progress reports of earnings at the end of their study (on top of averaged weekday earnings across time) since ``sometimes I'll have one [gig] a month, but then other times I won't have one \dots so it's hard for me to see patterns in my petsitting.''. For drivers like Driver-6 who wanted to show his friends evidence for or against working for Uber, he appreciated ways to ``compare \dots this is what I made last week \dots or a yearly thing: \dots [this is] the first week of September, and then this is what I made like last last year of the 1st week of September''
% pulling it out from a year ago to realize how much you know, gas we were paying from last year to this year


\paragraph{\textbf{``Downtime'' and ``Deadtime'': {Making} Total Work Time {Visible}}}
{Participants also suggested additional metrics to improve workers' understanding about time they spend working that they might sometimes overlook.}
To optimize working time, they explained the importance of including metrics and visualizations that illustrate not only of hours {actively booked on a job}, but also hours that are unaccounted for, such as ``downtime''---i.e., time spent waiting for work opportunities (Petsitter-4). 
%Petsitter-4 related how a Rover job that might claim to be a 30 minute walk can easily turn into a 50 minute job: ``\textcolor{red}{insert quote}''. She wanted a way to view her downtime compared to her uptime. 
% Michelle(petsitter) deadtime --- (for rover, it might factor more into travel time and also how do you organize your day if you're doing 4 walks...)
Drivers {also described wanting to record and view this ``downtime'' or in their case, a concept commonly referred to as ``dead time''}---time spent driving around {for rideshare} without a paying customer in the car.
Drivers-1 and -7 both thought it was imperative for drivers to know the proportions of their paid time within total working time, which includes the paid trip as well as time spent driving to pick up customers and waiting to receive a request between paid trips.
% Policy and literature around time for rideshare drivers has often labelled this as P1, P2, and P3 where (...).
%I do like to watch and what keep my hourly because I try to keep an average of anywhere between \$35 to \$42 an hour. (WHO SAID THIS) Kriz Driver-4? 3? I can't find it either



% Petsitter-1 imagined a simplified process for gathering overarching insights about her past earnings: ``it would be great to be able to pull data in 2023 \dots to see my income, the number of bookings I did --- an overview. So in the end I would be able to break it down a little bit easier, segregate out the amount of fees''.

% pulling it out from a year ago to realize how much, gas we were paying from last year to this year


 % michelle wanted mileage tracking, and then it'd have all she needed in terms of data it's collecting ("one-stop shop" to enter and view earnings, expenses, and mileage)
% [Driver-7 : Also wants to have miles tracked automatically like Stride app does for him.]
% [Driver-9 : Also wants b/c uses gridwise purely for mileage tracking rn. would recommend to all his friends if it had that b/c he can also track the rest--trips and expenses]
%And Tom also talked about how he's using multiple apps so having it all in one is preferable. 
% \textcolor{red}{-michelle mentioned something about different insights about trends of earnings for overnights vs drop in vs walks}


\subsubsection{{Ideas Around Additional Worker-Worker Interactions \& Anonymity Preferences}}
%subsubsection{Qualitative Data: Additional Story Interactions \& Anonymity} %angie to work on

\FloatBarrier
\begin{figure}[h!]
    \centering
\includegraphics[width=0.4\linewidth]{Chapters/images/p1_dogs.png}
    \caption{Petsitter-1's response to another strategy}
    \label{dogs}
\end{figure}
\FloatBarrier

\paragraph{\textbf{Commenting \& Reaction Options}}
%\paragraph{\textbf{Commenting (with Moderation), Integrated Uploads, Filtered Sharing, Reaction Options}}
% \textcolor{red}{P1 does not post on forums like reddit (around 30 mins) but ended sharing stories}
Currently, Stories have limited interactions: workers can post a story, read a story, or like a story. Participants held mixed feelings about new interactions{: while Petsitter-1 was adamant against implementing additional interactive (commenting) features ---``I don't think it's productive.'' --- most} expressed desires for comments and moderation mechanisms (Freelancer-1, Petsitter-3, Driver-2, Petsitter-5), 
.

---see Figure \ref{dogs} for one petsitter responding to another. Petsitter-4 {asked} about capabilities for networking with petsitters---she did this previously and finds it helpful to be on a list of trusted contacts that refer each other to clients. 
% \textcolor{blue}{sort out phrasing around here}

However, others wanted to prevent the Stories feed from duplicating behaviors on other forums such as subreddits or Facebook groups. Specifically, participants wanted to avoid off-topic posts or misunderstandings---"a lot of stuff gets taken out of context" (Driver-3). Petsitter-1 did not want further features added while others suggested limited interactions. Petsitter-4 suggested adding "Agree" and "Disagree" buttons. Driver-1 proposed that comments be allowed but only viewable by policymakers to mitigate worker disagreements: "let's say when a `Driver A' posts a story...I've dealt with the same issue, I comment on that story. The other drivers would not be able to see my comment. But the policymakers would be able to, which means no online arguments amongst drivers".
%
Driver-6 gave a unique idea to merge qualitative and quantitative data, {and improve readers' confidence} in the veracity of stories being shared. He uploaded Trip data and described an issue he faced with Upfront fare in the Notes field, and wanted to share this data with workers, policymakers, and advocates within the Story feed. He felt stories with real Trips attached could serve as proof or evidence that a story is not fabricated. On the other hand, Petsitter-5 did not believe pay information should be shared, but did encourage highlighting the photo and video sharing options more.

\paragraph{\textbf{Anonymity and An ``Edited'' Trail.}} 

Several participants requested more anonymity and traceability on the \textit{Story} feed, especially necessary once Gig2Gether is circulated and used more widely by different stakeholders. Though many participants were comfortable sharing stories with their usernames, others wanted strict anonymity, pointing out that they and others may use the same social media username across platforms. This could uncomfortably lead to being identified if peers of subreddit or Facebook groups join the platform and disagree with their stories (Petsitter-4). Driver-7 also wanted to share ``political'' stories but refrained because there was no way to post anonymously.

Drivers-1 and 6 both emphasized the importance of removing location information as well. Driver-1 asked: ``Is there any situation that would cause the trip data to be visible to other drivers? For instance, would they be able to see what city these are in at any time? \dots That's a very, very serious concern.'' Driver-6 advised for Gig2Gether to automatically blur sensitive information, such as addresses, that workers might upload, explaining drivers often forget this step when posting screenshots on forums which risks privacy and security. Driver-1 also highlighted the need to maintain an edit trail for posts. Users have different reasons for editing stories---a few participants said they appreciated the ability to go back and modify typos. However, he was concerned if deceitful users were to abuse the current lack of an edited label to gaslight others.
%driver-7 wanted anonymity, a political tag, that policymakers should view stories with the political tag first, for people to use this to collectively act against platforms to incite change, cared about undercut wages the most; 
This feedback underscored the importance of ensuring robust privacy controls within the application. 

\section{Discussion}

{Literature has surfaced what policy stakeholders feel gig worker data can be useful for---e.g., drafting bill language, communicating workers' platform experiences to colleagues \cite{policy_probes}. Research has also reported several initiatives and related data metrics that policy stakeholders desire for supporting worker-centered initiatives---e.g., aggregate data to investigate unfair pay or discrimination \cite{supporting}. Our study extends these insights with workers' initial impressions from using a data-sharing system intended to support individual goals and inform policymakers. First, we offer implications for how Gig2Gether could assist workers on supporting policy for \textit{worker safety} based on the stories participants shared. Then, we explain ongoing challenges for building a data-sharing system that ensures anonymous and truthful sharing. Finally, we reflect over how to ensure data-sharing on its own is not conflated as a catch-all solution by imagining how a system like Gig2Gether can complement worker empowerment and collective action efforts.

}
\subsection{Gig2Gether: ``A Door Between [Workers] and Politicians''} \label{door}

%%repeat, this is initial insights whereas we need to bring awareness to these efforts to really know the extent to which it can impact
Participants exhibited a weariness about whether actual policy or regulation would pass to improve their working conditions: ``Except for you guys, no one is trying to help us \dots no one is trying to expose any of the issues \dots If there's an issue with a passenger, the politicians are all over it. But with a driver, [not so much]'' (Driver-1). This skepticism for change is a natural barrier to the system reaching a critical mass of users before it can gain momentum in pushing forward policy initiatives.

However, participants were also \textbf{roused by the promise Gig2Gether represents as a vehicle for overcoming systematic challenges by enabling policymakers and advocacy groups access to data and issues workers face} for advancing change on labor {protections and initiatives}. {In fact, although participants} conveyed discouragement that policymakers are more invested in addressing consumer protections than worker issues, we observed several participants discussed platform issues by explaining the negative impact on both workers \textit{and} customers. For example, Petsitter-4 shared that Rover appears to be double charging fees on customers, and Driver-1 described how Uber's current verification policies puts drivers' and customers' safety at risk. \footnote{{We did not have a chance to ask if tying together worker and customer issues was unintentional or motivated by the belief this would garner strong policymaker attention.}} {We also recall} how Driver-7 suggested a new tag, ``Political,'' for workers to use to prioritize which stories policymakers should see first.
 and explicit sharing settings that ask workers to share the information they just posted with policymakers and/or advocates could raise awareness for some participants about the potentials of policymaking and advocacy that can improve their working conditions.
recall how Driver-7 suggested a new tag, ``Political,'' for workers to use to prioritize which stories policymakers should see first, supporting workers' desire for such a portal between workers and policymakers and being heard.
%
{Though our field study period was limitated to one week, this is a nascent glimpse into how several participants used the platform to contribute policy-aligned narratives and suggestions. This leads us to wonder whether continued contributions to a data-sharing system can 1) gradually influence workers not usually inclined to participate in ``political'' initiatives, to re-frame their motivations for data contribution, and 2) foster cross-stakeholder collaborations for actionable policy goals.}

{Reflecting on workers' Stories also suggests to us Gig2Gether's potential as a mechanism to influence policy protecting gig worker safety---the top concern shared by workers in Stories---in meaningful ways. Most gig-worker legislation and regulations in the U.S. are on minimum pay standards (e.g., NYC, Chicago, CA, WA, MN, MA), and more recently data transparency (e.g., CO). Yet little exists for worker safety. Colorado's recent data transparency bill contains language around delivery worker safety, though the extent appears to require platforms to send the customer a nudge ``to encourage the consumer to ensure driver safety upon arrival, including by ensuring a clear, well-lit, safe delivery path and ensuring all pets are properly secured.'' \cite{Colorado_General_Assembly_dnc_2024}. %turn into a citation, angie
One of the more meaningful attempts was NYC's regulations for food delivery workers' physical health and safety: the text explicitly grants workers access to restaurant bathrooms and allows them to ``set limits on travel over bridges or through tunnels and the distance between a restaurant and a customer''.\footnote{https://www.nyc.gov/site/dca/workers/workersrights/food-delivery-worker-laws-faqs.page} 
Participants' stories on Gig2Gether were rich and \textit{specific}, describing tactics for customers trespassing property (Driver-1), experiences breaking up animal fights (Petsitter-1), clients violating Terms of Service (Freelancer-1). These were all serious scenarios that workers were unsurprised by --- given their lived experiences --- but are likely non-obvious for policymakers or the public at large. Taking inspiration from the example set by NYC's delivery worker protections, we propose workers' stories on Gig2Gether could be leveraged in bill writing to inform language and requirements that meaningfully protect workers' physical and mental well-being.}

\subsection{Ongoing Challenges of Stakeholder Verification: Ensuring Anonymous, yet Truthful Sharing} \label{ongoing}

A data-sharing system that provides collective insights for different stakeholder groups must ensure both worker privacy (i.e., their data and identity in case of platform retaliation) as well as data reliability (i.e., that the contributors are real workers and the data is true and complete). Our pilots and field study were invite-only which allowed us to control access to only participants verified as qualified workers for our study. However, even with this closed system, we reflect on some challenges of securing a data-sharing platform faced by Gig2Gether and advocates or grass-roots organizations pursuing the same mission of data-driven insights towards worker well-being. 

First, while building the system, we were unable to find platform APIs to link gig worker accounts and validate both worker identity and completeness of their platform data uploads. As noted by \citet{dubal2023algorithmic}, third party data connectivity services exist \footnote{https://argyle.com/}, however these are expensive and have questionable data privacy and protection practices. Additionally, as our participants raised, manual entry would still be necessary for workers who accept off-app work. As discussed by Hsieh et al. \cite{supporting}, workers may take work off-app to make up for low wages on-app, which would be important for policymakers to be aware of to determine whether current gig work platforms are creating untenable wage conditions. %(https://home.coworker.org/wp-content/uploads/2021/11/Little-Tech-Is-Coming-for-Workers.pdf, page 31/73 on argyle). 
During our field evaluation, we verified workers by asking them to share their worker-side app profiles over video during the Zoom onboarding session, but this process is not scalable. Outside of APIs or data portability services, there are limited fool-proof document-based methods to verify workers. For example, Upwork subreddit threads demonstrate this as an ongoing challenge even for clients trying to verify freelancers to hire. Finally, if Uber drivers could share the CSV files that they download from the Uber platform, this could be a more reliable method of verifying their identity and data because it would be difficult to fabricate such data. However, the other platforms do not have this data download as an option for workers and we did not want to put Uber participants at risk of deactivation by requiring them to upload this file a pre-requisite for participation in the study. %This does point to a wider issue around data rights in the United States though...

This leads to a related question for a publicly deployed system around the veracity of information. The challenges of affordable and accessible methods of data portability for participants to easily connect or upload their work data from different platforms will invariably affect the quality of collective insights and stories shared through the system as well as workers' trust in the data.
%but also advocacy and policymakers (unless the system is closed to only workers of advocacy groups that they can choose the subset of---much like Driveres' Co-op did). 
One potential way of mitigating this for the Stories feed was suggested by Driver-6: Stories could be shared with related task-level data (e.g., a trip) as supporting evidence. %Relatedly, there could be affordances on collective insights to let 

{One caveat for data-sharing tools is \citet{khovanskaya2020bottom}'s warning that creating new data tools, especially ones to be managed by a union, can potentially burden union staff rather than enable change. Indeed, despite how workers entrusted us (researchers) with data, and Petsitter-1's thoughts about how advocates should act as intermediaries for bringing a data-sharing more to the awareness of workers, it remains unclear how to address issues of ownership. 
Harkening back to a collaborative model proposed in \cite{supporting}, one could imagine involving researchers in the management of technical maintenance of tools but using them in collaboration with unions. For instance, researchers might explore stories and metrics together with union representatives and workers to create membership recruitment material they desire for collective action.
}

%: For workers, seeing the contextualized stories that a worker shares might help workers  collectively deliberate whether a peer is a real ``worker''.

%Additionally, it raises the question about how advocacy groups and stakeholders can be verified. We can close the system to not allow anyone to register as a policymaker or advocate until we have met with them and create unique profiles for them to use, but it does create a barrier in case legitimate advocacy groups or policymakers wish to view it prior to reaching out to us to establish a connection/trust.  But typically, non-worker stakeholders will have public profiles anyways so we can very their identites thru government emails or specific organizational emails.   Workers remains an outstanding challenge. 

% Additionally, as it was a field study that sought to test both usability and feasibility of a data-sharing system, we did not employ a verification process for the manual data uploads workers provided. However, in the future, a lack of robust verification could risk the identities and ability to work for users of the Gig2Gether, and lead to hidden biases within the collective insights based on workers' selective sharing. 

% \begin{itemize}
%     \item liability of sharing on here ?? participant drop out?? Driver-4 dropped out after speaking with other s and worrying that sharing his driving data on Gig2Gether would violate his uber driver terms and conditions
%     \item incentivizing story sharing/sharing in general
%     \item concerns about how to verify identities that people are true workers or policymakers/advocacy groups again 1) people who are spammers, 2) people who work for the platforms themselves; and to verify the information they share is true. verification of worker identity is challenging depending on the platform
% \end{itemize}

%\subsection{Ownership, governance and moderation of Gig2Gether}
\subsection{{Data-sharing as a Complement to Alternative Methods of Worker Empowerment}}\label{complement}
{We reflect that relying on data and policy as a standalone, catch-all solution should be cautioned against. Instead, we encourage researchers to consider creative alternatives and worker-driven objectives that would benefit if combined with data-sharing. We share two suggestions motivated by the different ways participants used or imagined using Gig2Gether, as well as prior research on worker empowerment and collective action.}

{
\subsubsection{Informal Support Networks \& Mutual Aid.} Past work highlights the strengths of gig workers creating informal networks for mutual aid around purposes of companionship \cite{qadri2021s, atom}, pooling financial resources \cite{gray2019ghost, seetharaman2021delivery}, as well as sensemaking and strategy sharing \cite{mohlmannn2023algorithm}. These efforts can and do exist outside of a data-sharing system, and we do not wish to overlook other forms of assistance by overemphasizing a solutionist notion of data. Uniquely, participants' use of and ideas for Gig2Gether’s Stories feed suggests one way to complement worker mutual aid. 

First, Petsitter-4's inquiry about networking with other petsitters on Gig2Gether reminds us that not all gig workers have a built-in community to lean on. Public online forums like subreddits have low entry barriers for seeking peers, but provide limited social connection--- Reddit users are anonymous and do not undergo any verification. Meanwhile, mediums like WhatsApp groups and co-located gig workers can establish intimate connections, helping build necessary trust among workers for sharing mutual aid, but joining a group or finding peers to create one's own can be challenging. 

We recall that participants found solidarity and reassurance in reading others' \textit{Stories}, with some contrasting it as more productive and trustworthy than other forums due to its verified and ``more serious'' users. Promisingly then, we believe data-sharing systems like Gig2Gether could offer workers a different low-entry barrier option than Reddit that \textit{does} allow for verification. This verification could help workers feel more comfortable and trusting of one another more quickly (akin to local Whatsapp Groups), an important baseline for successful social bonding. In this way, systems like Gig2Gether can be leveraged to strengthen workers' abilities to build personalized networks for mutual aid.

%for workers to connect to one another and , even aligning future features with workers' desires to strengthen the potential for building personalized networks for mutual aid.

%By aligning future features with workers' desire to connect with each other, data-sharing systems like Gig2Gether can offer a low-entry barrier option that also ensures verification to help diverse and trusted workers connect and build personalized networks for mutual aid.

%How to help build closer networks that make mutual aid distribution feasible, form trusted contacts. 

%The second direction was the unexpected finding about how workers began to drum up interest in learning about other platforms from reading other platform work experiences on Stories. This highlights how it’s possible there may be other opportunities to match people’s interests and capabilities better but that they have not had the confidence to explore and learn. Support finding new opportunities in productive ways...

\subsubsection{Boosting Membership for Worker-Organizations.}
In the US, worker unions increasingly appear the news for activities related to collective bargaining (e.g., striking, calling for boycotts \cite{Yamat_2024, Robertson_2024, Reuters2024vw}) and policymaking (e.g., fighting against anti-union law or pushing for worker-centered laws \cite{Quinlan_2024a}). In fact, across the board, the National Labor Relations Board (NLRB) report on recent data revealing how union petitions (i.e., requests to unionize) and support for unions to be on the rise \cite{NLRB_2024}.

Worker-organizing for platform gig workers has also gained traction, especially with the NLRB's 2023 reversion to a worker classification standard that offers gig workers a way to join unions \cite{Cockayne_2023}. Yet, it is unclear to what extent this has influenced gig workers joining unions or the impact across different platforms. Literature suggests that challenges in unionizing gig workers remain---\citet{schou2023divided} found that differences such as motivations and identities can lead to conflicting goals and hinder attempts at collectivizing, despite shared outrage over worker issues (e.g., wages). As a counter to those differences, worker-organizers have expressed interest in presenting workers with their data in formats like data probes to help them identify platform manipulation they perceive and incense them into formally joining unions \cite{policy_probes} --- echoing Driver-7's desire to use Gig2Gether for showing specific collective insights to workers around low wages to encourage a strike. Increasing membership would boost a union's financial power to create change as member dues are crucial for unions to operate successfully---e.g., organize campaigns, negotiate and enforce contracts, provide training and legal assistance \cite{UnionCoded_2023}. 
}