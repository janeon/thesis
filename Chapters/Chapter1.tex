\chapter{Introduction}\label{ch:introduction}
Gig work, over the past decade, has driven transformative progress in both the online and offline labor markets. 
By expanding work access to previously marginalized individuals \cite{moving, disabled}, rapidly supplying labor demands just-in-time \cite{jit, making}, and attracting worker participation from around the globe \cite{woborders}, the platform-based gig economy prides itself in enabling the advent of alternative, more flexible forms of work. 
However, underlying such tremendous advancements are laborers who regularly contend with unprecedented risks and challenges: health risks and physical hazards \cite{body, health, technostress, protective}, intense competition and financial precarity \cite{dark, precarity, france, km}, over-surveillance and privacy violations \cite{surveillance, privacy}, a lack of social support and legal protections \cite{atom, category, employment}, not to mention algorithmically-amplified control \cite{locus, good}, gamification \cite{game, ludification}, and discrimination \cite{Leung2020-rk, Gelles-Watnick2021-fz, Rosenblat2017-bm}.

To exacerbate such conditions, platforms' refusal to disclose data that they (in)visibly collect on workers stifles progress in improving gig work conditions, despite the dual value that workers provide for platforms (as both service providers and data producers who are subject to platforms' surveillance-style data collection) \cite{dual}, platforms' core reliance on workers' personal data for powering their work assignment algorithms, workers' intention to engage in collective resistance \cite{boss}, calls by scholars across the world for increased legislation and regulatory action \cite{regulate, organizing}, or plans of governments to increase regulation \cite{deficit}. Combined with a lack of data governance in place, the reluctance of platforms to share information about workers forces policymakers into a data deficit \cite{deficit}, and access to such data by non-platform stakeholders (e.g., advocates, policymakers, workers themselves) remains out of sight. Moreover, platforms intentionally design their systems in a way that prevents workers from communicating or sharing work data with one another \cite{uuapp}, thereby intentionally perpetuating information and power asymmetries, as well as limiting worker abilities to engage in collective actions including group sensemaking, goal-setting, decision-making, and beyond. Such practices critically impedes advocacy progress for related advancements in labor regulation and gig worker classification, both of which already fall behind within the United States.

In order for workers to come together and push back on the power asymmetries imposed by platforms, it is imperative to address the current legal limitations that fall short in providing worker protections, as well as to construct digital infrastructures that can (1) facilitate resource exchange and peer support and (2) provide a means unifying and collectivizing workers towards achieving common goals and initiatives, such as petitions to change existing labor policy and regulations, as well as formulations of new structures for related governance. 

\textbf{Motivate research questions, individually or as groups}

This dissertation aims to identify both worker-centered tools and strategies for helping workers achieve individual success as well as ways of building collective power with peers, in a way that integrates into their existing workflows -- thereby narrowing the gap between HCI and policy approaches towards improving gig work conditions and its surrounding infrastructure. In particular, I sought to answer the following research questions:

\begin{itemize}
    \item RQ1: Can we glean successful gig work strategies by statistically analyzing data that workers contribute as a part of their labor on gig platforms? \cite{personal} 
    \item RQ2 (a): What methods of improving gig work conditions are most aligned with preferences of related stakeholder groups? \cite{alternatives}
    \item RQ2 (b): Where do federal labor protection laws fall short for gig work? \cite{individualized}
    \item RQ3 (a): How can datasharing drive policy initiatives of improving gig work conditions that engage both policy experts and affected workers? \cite{supporting}
    \item RQ3 (b): What system capabilities and interaction mechanisms do gig workers require to engage in information exchange around their labor and inform policy generation? \cite{gig2gether}
\end{itemize}

I approached this goal in three stages. In Chapter 1, I offer concrete, novel and data-driven insights about existing successful practices and strategies that online freelancers employ. In chapter 2, I engaged with multiple relevant stakeholder groups to shed light on points of alignment and contention with regards to how to address concerns around work conditions. In Chapters 3 and 4, my studies around data-sharing and policy innovations describe my process designing, developing and implementing new forms of technological support that empower, protect and unite platform-based laborers in their everyday work, while simultaneously informing relevant policy decisions. Future work will go beyond individual contribution mechanism for the data collective to explore interaction techniques that support geographically-dispersed workers in coordinating group decisions around relevant policy and advocacy efforts.
