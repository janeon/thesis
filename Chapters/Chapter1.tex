\chapter{Introduction}\label{ch:introduction}
Gig work, over the past decade, has driven transformative progress in both the online and offline labor markets. 
By expanding work access to previously marginalized individuals \cite{moving, disabled}, rapidly supplying labor demands just-in-time \cite{jit, making}, and attracting worker participation from around the globe \cite{woborders}, the platform-based gig economy prides itself in enabling the advent of alternative, more flexible forms of work. 
However, underlying such tremendous advancements are laborers who regularly contend with unprecedented risks and challenges: health risks and physical hazards \cite{body, health, technostress, Almoqbel2019-in}, intense competition and financial precarity \cite{dark, precarity, france, km}, over-surveillance and privacy violations \cite{surveillance, privacy}, a lack of social support and legal protections \cite{atom, category, employment}, not to mention algorithmically-amplified control \cite{locus, good}, gamification \cite{game, ludification}, and discrimination \cite{Leung2020-rk, Gelles-Watnick2021-fz, Rosenblat2017-bm}.

To exacerbate such conditions, platforms' refusal to disclose data that they (in)visibly collect on workers stifles progress in improving gig work conditions, despite the dual value that workers provide for platforms (as both service providers and data producers who are subject to platforms' surveillance-style data collection) \cite{dual}, platforms' core reliance on workers' personal data for powering their work assignment algorithms and workers' intention to engage in collective resistance \cite{boss}. In particular, the data deficit that workers experience (around their own personal data collected by platforms) hinders efforts to strategize and plan for optimized workflows or schedules. This barrier motivates the first investigation of this dissertation, which leveraged a freelance platform dataset to examine in Chapter \ref{freelance}:

\begin{enumerate}
    \item[\textbf{RQ1:}] Can we glean successful gig work strategies by statistically analyzing data that workers contribute as a part of their labor on gig platforms? \cite{personal} 
\end{enumerate}

Equipped with the understanding that large-scale data can indeed help workers uncover existing work patterns and strategies, I proceeded to investigate how to approach gig work futures where platforms can be held accountable to more just and transparent (data) practices. Up to this point, prior works primarily served to investigate and uncover the various forms of adverse conditions present in gig work. But beyond understanding the downstream harms that unregulated gig platforms inflict on workers, progress towards feasible and implementable solutions require input and conversations with multiple related stakeholders groups --- including those in higher positions of power than workers. Thus, in Chapter \ref{3codesign}, I engaged with multiple stakeholder groups to inquire:

\begin{enumerate}
    \item[\textbf{RQ2 (a)}] What methods of improving gig work conditions are most aligned with preferences of related stakeholder groups? \cite{codesign}
\end{enumerate}

One finding of this investigation corroborated the need for advancements in regulations of platforms, which scholars across the world have also been pushing for \cite{regulate, deficit, regulating, Dubal2017-bj, Dubal2019-qi}. In order to advocate for improvements in the legal and regulatory spaces, it is imperative to understand and address the current legal limitations that fall short in providing worker protections. Thus, in Chapter \ref{4individualized}, I synthesize exemplary cases where worker protections were recently encoded into certain state or city bills, as well as spaces where gig worker protections are insufficient when compared to standard employee rights --- to advocate for more individualized policy advancements:

\begin{enumerate}
    \item[\textbf{RQ2 (b):}] Where do federal labor protection laws fall short for individual gig workers? \cite{individualized}
\end{enumerate}

At the intersection of law and gig work, scholars across the world call for increased legislation and regulatory action \cite{regulate, organizing}.
However, the reluctance of platforms to share information about workers forces policymakers into a data deficit \cite{deficit}, and access to such data by non-platform stakeholders (e.g., advocates, policymakers, workers themselves) remains out of sight. 
Moreover, platforms intentionally design their systems in a way that prevents workers from communicating or sharing work data with one another \cite{uuapp}, thereby intentionally perpetuating information and power asymmetries, as well as limiting worker abilities to engage in collective actions including group sensemaking, goal-setting, decision-making, and beyond. To bridge this gap in datasharing between (and among) workers and policymakers, we must understand their (shared) initiatives of interest where specific worker data may be of assistance. In Chapter \ref{support}, I explore

\begin{enumerate}
    \item [\textbf{RQ3 (a)}]: How can datasharing drive policy initiatives of improving gig work conditions that engage both policy experts and affected workers? \cite{supporting}

\end{enumerate}

In order for workers to come together and push back on the power asymmetries imposed by platforms, they will require digital infrastructures that enable interactions with other workers (which is unfortunately not permitted by labor platforms) and potentially policymakers to (1) facilitate resource exchange and peer support and (2) provide a means unifying and collectivizing workers towards common goals, initiatives, and petitions to change existing labor policy and regulations. 

\begin{enumerate}
    \item [\textbf{RQ3 (b)}]: What system capabilities and interaction mechanisms do gig workers require to engage in information exchange around their labor and inform policy generation? \cite{gig2gether}
\end{enumerate}

In sum, this dissertation aims to identify strategies and worker-centered tools that support platform-based gig workers in both individual career advancement as well as in building collective power with peers. By grounding our findings on the existing lived experiences of workers, we approach this objective in a way that integrates into their workflows and addresses concrete needs that can be supported by furthered advances in policy and technology. \\ \\
\begin{tcolorbox}[colback=maroon!10,
colframe=maroon!60,
fonttitle=\bfseries,
fontupper=\normalsize\selectfont, 
% Adjust to match your document's text font
title=\textbf{Thesis Statement}]
\textit{By integrating insights and (data) needs of gig workers and policy experts, we can pursue digital infrastructures that more visibly document gig work conditions and strengthen worker communities, as well as policy changes that more effectively protect workers from power asymmetries  --- thereby propelling the gig workforce towards more thriving and sustainable futures.
}
\end{tcolorbox}