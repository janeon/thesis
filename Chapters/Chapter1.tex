\chapter{Introduction}\label{ch:introduction}
Gig work, over the past decade, has driven transformative progress in both the online and offline labor markets. 
By expanding work access to previously marginalized individuals \cite{moving, disability}, rapidly supplying labor demands just-in-time \cite{jit, making}, and attracting worker participation from around the globe \cite{woborders}, the platform-based gig economy prides itself in enabling the advent of alternative and more flexible forms of work. 
However, underlying such tremendous advancements are laborers who regularly contend with unprecedented risks and challenges: health risks and physical hazards \cite{body, healthdrive, technostress, jbho}, intense competition and financial precarity \cite{dark, precarity, france, km}, oversurveillance and privacy violations \cite{surveillance, privacy}, a lack of social support and legal protections \cite{atom, category, employment}, not to mention algorithmically amplified control \cite{locus, good}, gamification \cite{game, ludification}, and discrimination \cite{Leung2020-rk, Gelles-Watnick2021-fz, Rosenblat2017-bm}.

Exacerbating such conditions, platforms' refusal to disclose data that they (in)visibly collect creates boundaries that limit workers' information exchange with their peers and supporting stakeholders \cite{zhang24demystifying, peersupport}. 
Despite how workers dually provide both service and work data for platforms --- with the latter subjecting them to surveillance-style data collection \cite{dual} --- they remain outsiders to platform's collected data. 
Collectively, this data barrier stifles progress toward regulatory policy that improves gig work conditions \cite{regulating}. At individual levels, such information asymmetries hinder career advancement opportunities for each worker \cite{own}. 
% their intentions to engage in collective resistance \cite{boss}, and platforms' core reliance on such data to power their work assignment algorithms. 
In particular, the data deficit workers experience (around their own data) in platform-based work limits their abilities to strategize and plan for optimized workflows or schedules. This barrier motivates Part \ref{part1} of this dissertation, where we ask:

\begin{enumerate}
    \item[\textbf{RQ1:}] Can we glean successful gig work strategies by statistically analyzing data that workers contribute as a part of their labor on gig platforms? \cite{personal} 
\end{enumerate}

In Chapter \ref{freelance} I investigate this by quantitatively analyzing a large-scale dataset of communication data from workers (over two million messages, involving >56K projects and >58K freelancers) from a leading online freelancing site to investigate how standardizing and personalizing communication strategies associated with project level and career achievements. 
Results reveal how 1) curating (i.e., personalizing) bidding messages associated a worker's chances of winning the bid; 2) keeping standardized schedules when responding to client messages (i.e., avoiding instant replying during the day) correlated with higher likelihoods of completing a project; and 3) standardizing bidding text (e.g., writing templates to submit bids for multiple projects) related to freelancers earn revenue over the long term, since it enables them to bid to a higher volume of projects. This investigation demonstrated the feasibility of harnessing successful strategies that drive worker success using data available to gig platforms, thus answering RQ1 for the case of an online freelancing platform.

\vspace{1em}
\hrule
\vspace{1em}

Equipped with the understanding that large-scale data can indeed help workers uncover existing work patterns and strategies, I proceeded to explore (in Part \ref{part2})) the space of possible gig work futures where technology and policymaking can hold platforms accountable to more just and transparent (data) practices. Up to this point, recent research endeavors primarily uncovered the various forms of adverse conditions present in gig work, including labor exploitation through invisible logistical work \cite{regulate, capitalism, ming2024wage, cole2024wage}, amplifications of existing inequalities for lower-resourced groups \cite{participation, creativity, disruption, ma2022brush, Galperin2021-eh}, increased exposure to physical and health risks \cite{rc6G, body, Kerr_undated-zw,jbho,8xYL, geschwindt2022biking, rc6G,dCmn}, without the safeguards of workers' compensation or health insurance \cite{rc6G,xMHW}.

Beyond understandings on the downstream harms that result from the absence of regulation, impactful progress toward feasible and implementable solutions requires input and conversations with multiple related stakeholder groups, including those with power to affect decisions that advance gig work conditions. 
For example, stakeholders with influence on policymaking may be interested in considering ways to progress labor regulations through (state) legislative and administrative/enforcement processes, which lag behind litigation in this space \cite{regulate, collier2018disrupting, bernhardt2010broken}. Platform designers, as well, can directly affect future functionalities, operations, and services initiated by their employing platforms --- making them important groups to consult when considering tangible changes to the digital infrastructures that gig workers primarily interact with. Thus, in Chapter \ref{3codesign}, I engaged with multiple stakeholder groups to inquire:

\begin{enumerate}
    \item[\textbf{RQ2 (a)}] What interventions for gig work conditions are most aligned with the preferences of gig workers, policymakers and platform designers? \cite{codesign}
\end{enumerate}

Specific to advances in policymaking, I pondered:

\begin{enumerate}
    \item[\textbf{RQ2 (b):}] Where do federal labor protection laws fall short for individual gig workers? \cite{individualized}
\end{enumerate}

To explore the space of possible interventions, I began Chapter \ref{3codesign} by browsing related literature and news reports to inform the construction of scenarios that are illustrative of the issues prevalent in gig work. To identify feasible solutions to address these challenges, I then conducted co-design workshops with multiple stakeholder groups to pinpoint the advancements they are most motivated and poised to support and execute. In total, I conducted eight sessions with 7 local advocates / policymakers, 5 platform employees, and 8 gig workers, leveraging the speed dating design method to quickly elicit the preferences of stakeholders to address real-life gig work situations. 
In response to RQ2(a), I identified synergies and tensions for solutions between stakeholder groups, which included radical reimaginings of the existing public infrastructure and policy (e.g., universal healthcare, income pools coregulated by platforms and governments, worker-owned cooperatives), as well as more tangible and implementable but incremental interventions such as ways of enhancing work dispatching or helping workers connect with existing resources of the local workforce.

Related to policy, this investigation corroborated the need for advances in platform regulations, which scholars around the world have also advocated for \cite{regulate, deficit, regulating, Dubal2017-bj, Dubal2019-qi}. 
To further and expand the legal and regulatory attention surrounding gig work, it is imperative to understand and address current legal limitations that fall short to protect workers in their labor. Thus, in Chapter \ref{4individualized}, I synthesize exemplary cases where worker protections were recently encoded into certain state or city bills, as well as spaces where gig worker protections are insufficient when compared to standard employee rights --- to advocate for more individualized policy advancements. This closer examination of the existing policy landscape advocates for more targeted, individualized, and personalized policies, benefits, and protections --- as opposed to general, all encompassing solutions that neglect to account for diversity of gig task domains and the backgrounds of workers who complete this labor --- in hopes of more sustainably and scalably supporting platform-based workers. 
% In addition to providing an overview of the large variety of risks and work types present in platform-based labor, I offer recent and exemplary policy innovations from specific US localities (e.g., New York City, Washington, California, Massachusetts) that others might consider emulating, and also highlight gaps in federal policy and regulation where I envision future advancements, these include exemptions from antitrust laws, specialized policies to target particular work domains, as well as protections from discrimination and anti-retaliation.

\vspace{1em}
\hrule
\vspace{1em}

At the intersection of law and gig work, scholars across the world call for increased legislation and regulatory action \cite{regulate, organizing}.
But despite the potential of large-scale worker data to reveal key insights about gig work conditions and practices (unveiled in Chapter \ref{freelance}), the lack of access to platform data forces policymakers (among other related actors -- e.g., advocates, policymakers, workers themselves) into a data deficit \cite{deficit}. 
Moreover, platforms intentionally design their systems in a way that prevents workers from communicating or sharing work data with each other \cite{uuapp} --- intentionally perpetuating information and power asymmetries and limiting workers' abilities to engage in collective actions (e.g., group sensemaking and decision-making, goal-setting). 
In Part \ref{part3}, I seek to reduce these barriers to information exchange between stakeholders and among workers through the design and development of a data-sharing system to facilitate collectivism and inform policy. To start, I expand our understandings of the (shared) initiatives of interest where specific worker data may be of assistance to both impacted workers and
stakeholders with influence in policymaking. Thus, I first explore 

\begin{enumerate}
    \item [\textbf{RQ3 (a)}]: How can data-sharing drive policy initiatives (of interest to both policy experts and affected workers) to improve gig work conditions? \cite{supporting}
\end{enumerate}

\noindent After identifying aligned initiatives of interest, we can begin exploring practical system mechanisms and interactions.

\begin{enumerate}
    \item [\textbf{RQ3 (b)}]: What system capabilities and interaction mechanisms do gig workers require to engage in information exchange around their labor and to inform policy generation? \cite{gig2gether}
\end{enumerate}

To approach RQ3a, I explore in Chapter \ref{support} the preferences and requirements that policy experts and impacted workers held around the design of a data-sharing system to advance peer support and regulatory infrastructures. In particular, I collaborated with 11 policy experts in interviews and 14 workers of four task domains (freelancing, ridesharing, food delivery, pet sitting) in co-design workshops to explore initiatives of interest that the aggregated data may support, as well as preferred methods of aggregating and sharing worker data among one another and with policymakers. By engaging with both groups, I identified several shared desires for initiatives on advancing gig work conditions that data-sharing can support, including further protections of equity, safety, and fair pay, as well as an improved understanding of the algorithms that assign work and ratings. 

These codesign results revealed how workers (data producers) are willing to share labor data to advance regulations that improve working conditions with policymaking experts/influencers (data receivers) on topics of interest. But to collectively push back against platform-imposed power asymmetries, workers require advances in digital infrastructures \cite{uuapp, policy_probes, end} that open up possibilities of information exchange with other workers on the same platform, gig workers from other platforms, as well as advocating groups that support their labor --- e.g., labor organizers, policy experts. Although the idea of a cross-stakeholder and cross-platform data-sharing system holds potential for achieving several interventions identified in Chapter \ref{3codesign} (e.g., resource exchange / pooling, peer support, collective actions), both stakeholder groups raised several practical concerns in Chapter \ref{support} that can cause them to withhold data: trust, privacy, ownership, lack of accommodations for work diversity.

% that enable interactions with other workers (which is unfortunately not permitted by labor platforms) and potentially policymakers to (1) facilitate resource exchange and peer support and (2) provide a means unifying and collectivizing workers towards common goals, initiatives, and petitions to change existing labor policy and regulations. 

In Chapter \ref{6gig2gether}, I considered both the identified policy initiatives (that benefit from data collectives) and the design concerns / recommendations of workers and policy domain experts to create wireframes of a prototype data-sharing system. With active gig workers in its target domains, I iteratively refined its capabilities through pilot tests \cite{gig2gether}. 
This process resulted in \href{https://gigshare.web.app/}{Gig2Gether}, a web app with functionalities for workers to 1) track and share work data with each other, as well as 2) present such aggregate statistics and stories as evidence and motivation for policymakers and advocates to address pressing issues of gig work conditions. Through a 7-day field evaluation with 16 workers from three domains, I found that Gig2Gether facilitated cross-platform mutual support, enabled financial reflection and planning, and helped workers to envision future uses cases -- e.g., collaborative examinations of algorithmic speculations, informing policy on issues of safety and pay -- which motivated (latent) desiderata of additional interactive capabilities and data metrics.

In the proposed work (Chapter \ref{proposal}), I outline plans to co-design (counter-)data production mechanisms while aligning the policy objectives of both workers and policy experts. In particular, I propose a multistage process that engages both stakeholder groups in co-deliberation to identify methods of data visualization, cross-stakeholder interactions, and governance for collective data-sharing systems, using possible extensions of Gig2Gether as a boundary object. I began with a review of possible existing mechanisms from the system-building literature and follow-up to describe steps of the protocol for the iterative and stakeholder-centered development process.

In sum, this dissertation aims to identify strategies and worker-centered tools that support platform-based gig workers both in individual career advancement and in building collective power with peers. By focusing our findings on the existing lived experiences of workers, we approach this objective in a way that integrates into their workflows and addresses concrete needs that can be supported by further advances in technology and policy. \\ \\
\begin{tcolorbox}[colback=maroon!10,
colframe=maroon!60,
fonttitle=\bfseries,
fontupper=\normalsize\selectfont, 
% Adjust to match your document's text font
title=\textbf{Thesis Statement}]
\textit{By integrating experiences, insights and (data) needs of gig workers and policy experts, we can design and develop technological interventions to better align (policy) preferences, thereby uniting worker communities to engage in mutual support and collective actions that challenge existing work conditions, visibilize hidden labor and inform more effective regulatory policy.
% By integrating insights and (data) needs of gig workers and policy experts, we can pursue digital infrastructures that more visibly document gig work conditions and strengthen worker communities, as well as policy changes that more effectively protect workers from power asymmetries  --- thereby propelling the gig workforce towards more thriving and sustainable futures.
}
\end{tcolorbox}