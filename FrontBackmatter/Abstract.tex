%*******************************************************
% Abstract
%*******************************************************
%\renewcommand{\abstractname}{Abstract}
\pdfbookmark[1]{Abstract}{Abstract}
\begingroup
\let\clearpage\relax
\let\cleardoublepage\relax
\let\cleardoublepage\relax

\chapter*{Abstract}
Gig work, as an alternative form of employment, has drastically transformed how modern society labors, hires and transits within the past decade and half. But while gig work affords consumers extensive conveniences and support,
laborers themselves face unprecedented and often unobserved physical, psychological, and financial challenges. 
This dissertation focuses on advancing technological and policy solutions that (1) empower and protect gig workers from the harms of data in-transparency and algorithmic management, which contribute to discrimination, over-surveillance as well as compromises of wages and safety, while (2) unifying worker communities and general public understanding of working conditions to resist current shortcomings in labor policy, regulation, and worker classification. 

First, I present successful strategies that online freelancers employ, resulting from a quantitative analysis of worker messages from a leading online freelancing site, showing how personalization and standardization associated with success factors like job acquisition, project completion and long-term revenue. Beyond existing worker strategies, I also describe our design exploration of how related stakeholder groups envisioned (individualized) improvements practical but persistent issues plaguing gig workers. In particular, I showed stakeholders (local advocates/policymakers, platform employees and gig workers across sectors) compelling scenarios of gig work issues based on real-world cases documented in the literature and the press to workshop potential solutions and uncover latent desires and fears. 

Next, I overview our development of a prototype data-sharing system, designed to advance solidarity, information exchange and related policy decisions. Data collectives embody one form of technological innovation to facilitate worker collectivism, advance advocacy, and inform policy. Collaborating with policy domain experts and workers to codesign this system -- I identified data initiatives of interest between the groups (e.g., equity, safety, fair pay) as well as shared concerns and visions around data privacy and ownership. These design objectives informed Gig2Gether, a web app allowing workers across platforms to exchange stories, track and share work data, and present aggregated statistics and evidence to policymakers and advocates. Our week-long field study with 16 workers showed 1) them using the system to record aggregatable and qualitative data, 2) enthusiasm to share uploaded data with peers and policymakers and 3) yearnings for additional methods of data-sharing.

Finally, I demonstrate the potential for game (mechanisms) to engage a broader audience in advocating for improved gig work conditions. Focusing on the rideshare context, we explore the potential for gamified in-ride interactions to advance passengers' understanding, empathy and advocacy for underexposed rideshare driving conditions and driver vulnerabilities.
Through a series of workshops with 19 drivers and 15 riders, we revealed passenger knowledge gaps around rideshare vulnerabilities, tradeoffs and opportunities around consent and content in gamified in-ride interactions, as well as considerations of alternative interactions and incentives for achieving further awareness among the ridership.

\vfill

\endgroup			

\vfill