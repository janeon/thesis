%*******************************************************
% Abstract
%*******************************************************
%\renewcommand{\abstractname}{Abstract}
\pdfbookmark[1]{Abstract}{Abstract}
\begingroup
\let\clearpage\relax
\let\cleardoublepage\relax
\let\cleardoublepage\relax

\chapter*{Abstract}
Gig work as an alternative form of labor has transformed how modern society labors, hires and transits within the past decade. To support such tremendous progress, gig workers face unprecedented challenges physically, digitally, and financially. This dissertation focuses on advancing technological (and subsequently policy) solutions that empower, protect gig workers from data harms (e.g., discrimination, over-surveillance) and compromises of wages and safety, as well as unite them towards collective action to resist current shortcomings in labor policy, regulation, and worker classification. 


First, I present successful strategies that freelancing workers employ, resulting from a quantitative analysis of worker messaging from a leading online freelancing site to investigate how various approaches are associated with success factors (e.g., acquiring jobs, completing
projects, long-term revenue). Our results show that 1) personalizing bidding messages relate to winning a bid;
2) keeping schedules associated with project completion and 3) standardizing bidding text aligned with higher
revenue earned over the long term.

Then, I follow up with how related (including power-wielding) stakeholder groups envisioned individualized improvements for gig work conditions. To tackle practical but persistent issues that plague gig workers, we looked beyond existing worker strategies to solicit insights from multiple stakeholder groups. First, we gathered cases of commonly observed struggles
in gig work from related literature and news reports and presented example scenarios based off of these
realistic work conditions to 7 local advocates/policymakers, 5 platform employees and 8 gig workers to workshop potential solutions using the speed dating method. Our findings reveal calls for changes in the US public infrastructure, policies and labor regulations, as well as individualized and technology interventions.

Next, I overview our process designing, implementing and evaluating a prototype data-sharing system, intended to facilitate solidarity and information exchange, as well as affect related policy decisions. Data collectives embody one form of technological innovation to facilitate worker collectivism, advance advocacy, and inform policy. We collaborated with 11 policy domain experts and 14 workers across four task domains to envision and codesign such a data sharing system, identifying initiatives of interest between worker and policy domain expert participants (e.g., data for protecting equity, safety, fair pay, and furthering understanding of algorithms). In subsequent work, we built Gig2Gether, a web app with functionalities for workers to exchange stories, track and share work data, and present aggregated statistics for presenting evidence on pressing issues to policymakers and advocates. Our 7-day field study with 16 workers from three domains show how workers 1) are capable of using the system to record and exchange both qualitative and quantitative data, 2) are comfortable sharing their uploaded data with peers and policymakers and 3) yearn for additional methods of data collection.

For proposed work, I outline plans to align means of data production for making impacts on policy. In particular, I plan to conduct codesign workshops with policymakers and advocates explore 1) more effective ways of analyzing and visualizing collective data to maximize the system's impact in informing relevant decisions and initiatives, 2) novel ways of cross-worker and cross-stakeholder interactions, as well as 3) feasible ownership, governance and moderation mechanisms to ensure both data quality and integrity, along with incentivized and sustained worker data-sharing.
% with the aim of producing a policy memo/brief to summarize key findings from this data collection effort. 
% Finally, I outline critical challenges remaining in the design and extension of such existing datasharing efforts, including requirements for building worker trust, ownership and incentives for contribution, establishing governance and moderation structures, as well as addressing lingering concerns around privacy, security and equity. 
% Finally, we consider how similar data collective effort can assist workers of adjacent domains (e.g., small businesses) and political contexts outside of the US.

\vfill

\endgroup			

\vfill