%*******************************************************
% Abstract
%*******************************************************
%\renewcommand{\abstractname}{Abstract}
\pdfbookmark[1]{Abstract}{Abstract}
\begingroup
\let\clearpage\relax
\let\cleardoublepage\relax
\let\cleardoublepage\relax

\chapter*{Abstract}
Gig work, as an alternative form of employment, has drastically transformed how modern society labors, hires and transits within the past decade and half. But while gig work affords consumers extensive conveniences and support,
laborers themselves face unprecedented and often unobserved physical, psychological, and financial challenges. 
This dissertation focuses on advancing technological and policy solutions that (1) empower and protect gig workers from the harms of data in-transparency and algorithmic management, which contribute to discrimination, over-surveillance as well as compromises of wages and safety, while (2) unifying worker communities and general public understanding of working conditions to resist current shortcomings in labor policy, regulation, and worker classification. 
% Our results show that 1) personalizing bidding messages relate to winning a bid;
% 2) keeping schedules associated with project completion and 3) standardizing bidding text aligned with higher
% revenue earned over the long term.

First, I present successful strategies that online freelancers employ, resulting from a quantitative analysis of worker messages from a leading online freelancing site, showing how personalization and standardization associated with success factors like job acquisition, project completion and long-term revenue. Beyond existing worker strategies, I also describe our design exploration of how related stakeholder groups envisioned (individualized) improvements practical but persistent issues plaguing gig workers. In particular, I showed stakeholders (local advocates/policymakers, platform employees and gig workers across sectors) compelling scenarios of gig work issues based on real-world cases documented in the literature and the press to workshop potential solutions and uncover latent desires and fears. 
% Our findings call for changes in the US public infrastructure, policies and labor regulations, as well as individualized and technology interventions.

Next, I overview our development of a prototype data-sharing system, designed to advance solidarity, information exchange and related policy decisions. Data collectives embody one form of technological innovation to facilitate worker collectivism, advance advocacy, and inform policy. Collaborating with policy domain experts and workers to codesign this system -- I identified data initiatives of interest between the groups (e.g., equity, safety, fair pay) as well as shared concerns and visions around data privacy and ownership. These design objectives informed Gig2Gether, a web app allowing workers across platforms to exchange stories, track and share work data, and present aggregated statistics and evidence to policymakers and advocates. Our week-long field study with 16 workers showed 1) them using the system to record aggregatable and qualitative data, 2) enthusiasm to share uploaded data with peers and policymakers and 3) yearnings for additional methods of data-sharing.

Finally, I demonstrate the potential for game (mechanisms) to engage a broader audience in advocating for improved gig work conditions. Focusing on the rideshare context, 
% However, the marketed consumer affordances give rise to financial, logistical and emotional burdens, which are often managed by drivers alone. 
we offer a design exploration of how persuasive game design mechanisms can promote consumer engagement and oversight around driver protections and conditions, motivating more accurate and transformed perceptions of what ridesharing looks like for drivers. 
In this study, we explore through a series of workshops with 6 drivers and 10-15 riders to examine how variours designs of casual games may advance passengers' understanding of rideshare driving conditions.
We uncovered the potential for ZZ, preferences for AA and tensions for BB.


% plans to align means of data production for making impacts on policy. In particular, I plan to conduct codesign workshops with policymakers and advocates explore 1) more effective ways of analyzing and visualizing collective data to maximize the system's impact in informing relevant decisions and initiatives, 2) novel ways of cross-worker and cross-stakeholder interactions, as well as 3) feasible ownership, governance and moderation mechanisms to ensure both data quality and integrity, along with incentivized and sustained worker data-sharing.
% with the aim of producing a policy memo/brief to summarize key findings from this data collection effort. 
% Finally, I outline critical challenges remaining in the design and extension of such existing datasharing efforts, including requirements for building worker trust, ownership and incentives for contribution, establishing governance and moderation structures, as well as addressing lingering concerns around privacy, security and equity. 
% Finally, we consider how similar data collective effort can assist workers of adjacent domains (e.g., small businesses) and political contexts outside of the US.

\vfill

\endgroup			

\vfill